\chapter{Stand der Technik}
\cite{Carabin.2017} gibt einen Überblick über bestehende Methoden zur Verbesserung der Energieeffizienz von Industrierobotern (IR) durch Anpassung der Hard- und/oder Software. Hardwareanpassungen werden in dieser Arbeit nicht untersucht. Aus Gründen der Vollständigkeit werden sie an dieser Stelle zusammenfassend dargestellt. Ein erster Ansatz basiert auf konstruktiven Änderungen an Bauteilen, z. B. dem Austausch schwerer Robotergreifer durch Leichtbaukomponenten. Durch die Reduzierung von Gewicht und Massenträgheit wird das aufzuwendende Drehmoment in den Antrieben des Roboters minimiert. Ein zweiter Ansatz ist die Installation von Systemen oder Komponenten zur Energierückgewinnung im Antriebsstrang während der Bremsphase. \cite{Pellicciari.2015} erarbeitet ein Konzept für den Austausch der zurückgewonnenen Energie über ein Gleichstromnetz (DC-Netz). Dabei wird unter anderem die DC-Netzanbindung des Roboters über eine speziell entwickelte Umrichter-Schnittstelle für den bidirektionalen Energiefluss skizziert.
%
Einen entscheidenden Beitrag zur Ausarbeitung Software gestützter Ansätze zur Energieverbrauchsminimierung leistet \cite{Eggers.2019}. Dabei werden Grundsätzlich zwei Konzepte unterschieden. Der Energiebedarf Taktzeit unkritischer Anlagen kann durch einen  Änderung der Verfahrzeit  minimiert werden. Der Pfad des Roboters wird nicht verändert. Im Kontext einer vollständigen Produktion stellt dies einen vielversprechender Ansatz dar, da das Kollisionsrisiko des Roboters bereits bei der Abnahme des bestehenden Programms betrachtet wurde. Eine Ausnahme bilden hierbei interagierende Roboter, da deren Bewegungsabläufe synchron anzupassen sind. Insbesondere in Phasen, in denen die Anlage auf einen vor- oder nachgelagerten Prozess wartet, kann eine Arbeitspunktbestimmung des Programm-Override (OR) erfolgen. Infolgedessen ist auch eine Anpassung der Bremseinschließzeit möglich. Der Ausarbeitung \cite{Eggers.2019} ist hinzuzufügen, dass eine Änderung der Geschwindigkeitsprofile je nach Bewegungsart Auswirkung auf die Trajektorie und damit den Pfad des Roboters nehmen kann. Ein zweiter Ansatz, der in \cite{Eggers.2019} untersucht wird ist die Optimierung der Bahngeometrie. Im Gegensatz zu früheren Ausarbeitungen wird der Ansatz in umfangreichen praktischen Szenarien validiert. 
%
% Modell
%
Hervorzuheben ist das in \cite{Eggers.2019} ausgearbeitete, erstmalig in \cite{Ziaukas.2017} publizierte und als Patent angemeldete Energiemodell \cite{Patent.2016} für die Antriebe des IR. Im Gegensatz zu \cite{Pellicciari.2011}, \cite{Sergaki.2002} und \cite{Paryanto.2015} werden nicht nur die Verluste der elektrischen Komponenten betrachtet, sondern zusätzlich die Betriebszustände des IR in die Phase MOTION (Roboter ist in Bewegung), HOLD (Roboter im Stillstand, Antriebe sind in Regelung) und IDLE (Haltebremsen sind aktiv) unterschieden und der daraus resultierende Energiebedarf analysiert\cite{Ziaukas.2017}. 
%Im Rahmen einer Ausarbeitung zur Optimierung von Industrierobotern für Hochgeschwindigkeitsanwendungen demonstriert \cite{Gattringer.2013} die praktische Umsetzung der in \cite{Siciliano.2011} beschriebenen System Identifikation.
%
%Optimierer
%
Auf der Grundlage des Modells wird der Energieverbrauch entlang einert Bewegungsbahn in \Cite{Hansen.2012} mit einem Gradienten Abstiegsverfahren minimiert. \cite{Ziaukas.2017} nutzt für das selbe Ziel einen active-set Algorithmus. Für die theoretischen Grundlagen der Nichtlinearen-Optimierung wird auf \cite{Nocedal.2006}, \Cite{Papageorgiou.2015} und \cite{Luenberger.2021} verwiesen. In \cite[S.10~ff.]{Carabin.2017} wird auf alternative Algorithmen hingewiesen. Ausgangspunkt der numerischen Optimierung sind  die Definition eines Optimierungsproblems und Aufstellung einer Zielfunktion.  Im Mittelpunkt stehen dabei Ansätze zur Reduktion des Energieverbrauchs  \cite{Ziaukas.2017} mit einer Zielfunktion zur Minimierung der auftretenden Motordrehmomente des IR. Alternativ wird die, vom Roboter aufgenommene DC-Netzleistung als Zielfunktion herangezogen \cite{Hansen.2012}. Neben der Definition des Optimierungsproblems einer Reduktion des Energieverbrauches wird die Minimierung der Verfahrzeit zur Reduktion der auftretenden Drehmomente in \cite{Pellicciari.2011} und \cite{Ziaukas.2017} angestrebt. \cite{Lin.2018} schlägt eine Mehrzieloptimierung über den  Energieverbrauch und die Verfahrdauer vor. Die häufig in den Grundlagen zitierte Ausarbeitung \cite{Saravanan.2008} betrachtet in der Optimierung simultan die Minimierung der vom Roboter aufgenommenen Leistung, den Ruck  und die Beschleunigung der einzelnen Gelenke. Abschließend sei die Ausarbeitung  \cite{Bjorkenstam.2013} genannt, welche an Stelle der nichtlinearen Optimierung die Theorie der optimalen Steuerung für eine Energieeffiziente und Kollisionsfreie Roboterbewegung zu Grunde legt.
%
%Trajektorie
%
Eine Notwendigkeit für die Optimierung einer Roboter-Bewegungsbahn ist die Definition der Bahnplanung. Hierbei werden drei wesentliche Ansätze unterschieden. In \cite{Hansen.2012} wird die Point-to-Point (PTP) Bewegungsbahn über eine B-Spline Funktion definiert. Entscheidender Nachteil dabei ist die Übertragung der Funktion auf eine industrielle Robotersteuerung aufgrund fehlenden Entwickler Schnittstellen zur Vorgabe von Sollwerten \cite[S.~55~f.]{Eggers.2019}. Eine praktikablere Umsetzung bietet die Definition und Verschiebung von zusätzlichen Via-Punkten \cite[S.~261~ff.]{Spong.2020}. Nach der Identifikation einer energieoptimierten Gelenkwinkel-Definition wird der Via-Punkt auf die Robotersteuerung übertragen. Der vom Hersteller der Robotersteuerungen implementierte Bahnplanungsansatz wird als nicht bekannt vorausgesetzt. Infolgedessen sind Abweichungen der von der Robotersteuerung berechneten Bahn gegenüber der optimierten Bewegungsbahn möglich. \cite{Eggers.2019} vermeidet dieses Problem durch die Berechnung der zu optimierenden Bewegungsbahn auf der Originalsteuerung in einem Software-in-the-Loop (SiL) Ansatz. Gleichzeitig werden damit Nebenbedingungen wie eine Drehmoment Begrenzung in der Antrieben berücksichtigt. 
%
Die vorliegende Arbeit verfolgt nicht das Ziel, dieselbe Modellgenauigkeit durch eine Parameteridentifikation zu erzielen wie \cite{Pellicciari.2011} und \cite{Gattringer.2013} oder das Betriebsverhalten der Antriebe nachzubilden wie \cite{Eggers.2019} bzw. \cite{Ziaukas.2017}. Vielmehr liegt der Fokus darauf die Grundlagen der Modellierung und Optimierung darzulegen, welche in der genannten Literatur als als bekannt vorausgesetzt und nicht näher ausgearbeitet werden. 
%todo einzelne Kapitel beschreiben
Des Weiteren sind zu erzielende Optimierungsergebnisse durch eine praktische Untersuchung zu belegen. In diesem Rahmen erfolgt zudem eine Vorbereitung der Schnittstelle zur Signalaufzeichung von Bewegungsdaten am IR. Da ein Interface zur Anwendung eines SiL-Ansatz nicht vorhanden ist, werden die Trajektorien der Gelenkwinkel über ein Polynom sechster Ordnung geplant. 