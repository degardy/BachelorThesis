\chapter{Validierung des Roboterdynamik-Modells}
%
Unter Berücksichtigung der getroffenen Annahmen wird untersucht, ob das implementierte Modell die Roboterdynamik hinreichend genau simuliert. Dafür wird der simulierte Drehmomentverlauf qualitativ und quantitativ mit den gemessenen Werten verglichen
%
\section{Messaufbau und Steuerungsarchitektur}
Als Testumgebung dient eine Roboterzelle in der digitalen Anlauffabrik im Mercedes-Benz Werk Berlin.
Die Roboterzelle stellt einen von acht Laborbereichen zur Erprobung standardisierter Fertigungstechnologien dar.  Der abgegrenzte Zellbereich ist mit mit Sicherheitseinrichtungen zur Personen- und Maschinensicherheit ausgestattet, sodass ein erheblich minimiertes Unfallrisiko in den Praxisversuchen gewährleistet ist. Der Roboter besitzt keinen Endeffektor (EE), so dass das Kollisionsrisiko in den Versuchen im Vergleich zu EE mit komplexen Geometrien minimiert wird. Darüber hinaus ist eine signifikante Abweichung von der ursprünglichen Bahn während der Bahnoptimierung nicht zu erwarten, da die Robotersteuerung in der PTP-Bewegungsart den minimalen Weg im Gelenkraum berechnet. Eine stark abweichende Bahn im kartesischen Raum würde zu einem längeren Weg im Gelenkraum und damit zu einem höheren Energieverbrauch führen.~\cite[S.~59]{Eggers.2019}
%
Ein Arbeitsgebiet in der digitalen Anlauffabrik ist die Auswertung von prozessbezogenen Daten, sodass der Roboter über Schnittstellen zum Datenabgriff interner Messeinrichtungen verfügt. Der installierte KUKA KR 210 wird über eine KR C5 gesteuert, welche die KUKA RobotSensorInterface 5.0 (RSI) Technologie unterstützt. RSI ist eine Schnittstelle zur zyklischen Kommunikation zwischen dem Industrieroboter und einem Sensorsystem. Die Robotersteuerung kommuniziert mit dem Sensorsystem über eine echtzeitfähige Netzwerkverbindung. Die Daten werden dabei über das Ethernet UDP/IP-Protokoll \footnote{User Datagram Protocol (UDP) ist ein verbindungsloses Protokoll für den Datenaustausch von Netzwerkteilnehmern~\cite{RSI.2020}.} übertragen ~\cite[S.~11]{RSI.2020}. 
Auf der Robotersteuerung ist daf{\"u}r ein RSI-Kontext zu parametrieren, wobei festgelegt wird, welche Daten an eine festgelegte Ziel-IP-Adresse inkl. Port {\"u}bertragen werden.~\cite[S.~43]{RSI.2020}.  Der Datenaustausch findet unidirektional zwischen zwei Netzwerkteilnehmern statt. Die KR C5 hat die Funktion eines sendenden Clients. Der Empfänger (Notebook) stellt den Server dar. Auf dem Empfänger wird die Ethernet Verbindung mit der definierten IP Adresse eingerichtet. Des Weiteren wird der im RSI Kontext festgelegte UDP-Port freigegeben. Der Port-Zugriff wird auf dem Windows Betriebssystem über die Programmierschnittstelle (Application Programming Interface(API)) Windows Sockets 2 (Winsock) realisiert \cite{Winsock.2023}. 
%
Der entsprechende RSI-Kontext wird im Initialisierungsschritt des KRL-Programms aktiviert ~\cite[S.~50]{RSI.2020}. Anschließend werden die Daten der erfassten Winkelverläufe und Drehmomente der einzelnen Gelenke in einem Takt von 4 ms in der American Standard Code for Information Interchange (ASCII) Darstellung an den Server übertragen. Die Daten werden in einem Textfile gespeichert und mithilfe von Python Skripten in die JavaScript Object Notation (JSON) konvertiert, verarbeitet und visualisiert. 
%
Neben den Winkelverläufen werden die Winkelgeschwindigkeiten, Winkelbeschleunigungen, sowie die mechanischen Leistungen der Gelenke benötigt. Da diese Daten nicht explizit zur Verfügung stehen, werden sie über den Differenzenquotient siehe Kapitel \ref{sec:tarjektorie} gebildet. Durch Multiplikation der Winkelgeschwindigkeiten und Drehmomente wird der zeitliche Verlauf der mechanischen Leistungen der Gelenke berechnet.

\section{Durchführung}
% Temperatur kalt und warm 
%Versuchsbeschreibung
% todo Beschreibe Durchführung

\section{Auswertung}
Zunächst werden die betrachteten Größen festgelegt. Die Auswertung basiert auf den Verläufen der Winkelgeschwindigkeit, welche in den Abbildungen \ref{fig:winkelgeschwindigkeit} und \ref{fig:winkelgeschwindigkeit_py1} dargestellt sind. Des Weiteren werden die simulierte Drehmomentverläufe in Abbildung \ref{fig:taumat} sowie die gemessenen Werte in Abbildung \ref{fig:tau} berücksichtigt. In beiden Fällen werden die Drehmomente auf der Getriebeseite abgebildet. Durch diese Darstellung werden die Zusammenhänge zwischen Drehmoment, Winkelgeschwindigkeit und mechanischer Leistung klar erkennbar. Eine Darstellung der Drehmomente auf der Motorseite wird vermieden. Durch Anwendung der Getriebe-Übersetzung ergibt sich eine Änderung des Vorzeichens der Momente in den Gelenken eins, zwei, vier und fünf, wie in Anhang \ref{add:systemparameter} angegeben, so dass die Zusammenhänge nicht ad hoc erkennbar sind. Die gemessenen Drehmomente werden in den Grenzen 0 s bis 1,3 s zu betrachtet. Daten für den Zeitraum $t>1,3\text{s}$ entfallen auf den Vorgang der Antriebsregelung bis zum Einfallen der Haltebremsen, um den Roboter in Position zu halten. 
%
\begin{figure}[tbph]
	\centering
	\includegraphics[width=1\linewidth]{images/taumat}
	\caption{Drehmomentverlauf simuliert ohne Gewichtsausgleich}
	\label{fig:taumat}
\end{figure}
%
\begin{figure}[tbph]
	\centering
	\includegraphics[width=1\linewidth]{images/tau}
	\caption{Drehmomentverlauf gemessen}
	\label{fig:tau}
\end{figure}
%
\subsection{Quantitative Auswertung Drehmoment}
Die simulierten Drehmomente fallen signifikant größer aus als die gemessenen. Der negative Maximalwert von $\tau_2$ liegt in der Simulation bei ca. -12 kNm. Dem gegenüber steht ein gemessener Wert  von ca. -2,8 kNm. Die Differenz wird der Vernachlässigung des Gewichtsausgleichszylinders zugeschrieben. Dieser kompensiert insbesondere den Einfluss der Gewichtskraft nachfolgender Glieder am zweiten Gelenk. Exemplarisch zeigt Abbildung \ref{fig:gewichtskraft} den simulierten Anteil der Gewichtskraft bei Stillstand des Roboters in der Ausgangsstellung $q_{s,i}$ siehe Tabelle \ref{tab:simu}. Für die Drehmomente am zweiten Gelenk ist erkennbar, dass der Anteil der Gewichtskraft von -7743 Nm genau dem  Offset des Drehmoments beim Start der Bewegung entspricht. In der Konsequenz fällt $\tau_2$ im Gütekriterium der Optimierung zu stark ins Gewicht. Daraus wird abgeleitet, dass der Gewichtsausgleich in der Modellbildung zu berücksichtigen ist. 
%
\begin{figure}[tbph]
	\centering
	\includegraphics[width=1\linewidth]{images/Gewichtskraft}
	\caption{Anteil der Gewichtskraft an den Drehmomenten in der Startposition}
	\label{fig:gewichtskraft}
\end{figure}
%
Idealerweise kompensiert der Gewichtsausgleich den Einfluss der Gewichtskraft aller nachfolgenden Glieder im zweiten Gelenk, siehe Abbildung \ref{fig:taumat-fgall}. 
%
\begin{figure}[tbph]
	\centering
	\includegraphics[width=1\linewidth]{images/taumat-fgall}
	\caption{Drehmomentverlauf simuliert inkl. idealen Gewichtsausgleich}
	\label{fig:taumat-fgall}
\end{figure}
%
Die Idealkompensation bildet jedoch nicht die Realität ab. Es wird ein Kompromiss gewählt, bei dem ausschließlich die  Gewichtskraft des zweiten und dritten Verbindungsglieds ausgeglichen wird. Das Roboter-Dynamikmodell wird um die Vorschrift \ref{eqn:submugrav} erweitert.
%
\begin{equation}
	\label{eqn:submugrav}
\bm{\mu}^{i}_{i} = \bm{\mu}^{i}_{i} - {\bm{\mu}^{i}_{i}}_{grav} ~\forall ~i \in \{2\}, 
\end{equation}
%
wobei
%
\begin{equation}
	\label{eqn:mugrav}
	{\bm{\mu}^{i}_{i}}_{grav} = -{\bm{f}^{i}_{i}}_{grav} \times \left( \bm{r}^{i}_{i-1,i} + \bm{r}^{i}_{i,C_i} \right) + \bm{R}^{i}_{i+1} {\bm{\mu}^{i+1}_{i+1}}_{grav} + \bm{R}^{i}_{i+1} {\bm{f}^{i+1}_{i+1}}_{grav} \times \bm{r}^{i}_{i,C_i} ~\forall ~i \in \{1,2,3,4,5,6\}
\end{equation}
%
und 
%
\begin{equation}
	\label{eqn:subfgrav}
	{\bm{f}^{i}_{i}}_{grav} = \bm{R}^{i}_{i+1} {\bm{f}^{i+1}_{i+1}}_{grav} + m_i\ddot{\bm{p}}^{i}_{C_i} - {\bm{R}^0_i}^T - m_i \bm{g} ~\forall ~i \in \{1,2,3,4,5,6\},
\end{equation}
%
wobei $\bm{g}$ die Fallbeschleunigung $[0~0-9,81]^T ~\frac{\text{m}}{s}^2$ ist.
Es folgt ein Drehmomentverlauf gemäß Abbildung \ref{fig:taumat-fg}. 
%
\begin{figure}[tbph]
	\centering
	\includegraphics[width=0.9\linewidth]{images/taumat-fg}
	\caption{Drehmomentverlauf simuliert inkl. näherungsweise modellierten Gewichtsausgleich}
	\label{fig:taumat-fg}
\end{figure}
%
% todo Quelle Reibung Getriebe
%
\subsection{Qualitative Auswertung Drehmoment}
Die Gelenkverläufe werden einzeln betrachtet. Für das erste Gelenk weisen die gemessenen Werte einen steileren Anstieg auf. Dieser Anstieg ist äquivalent zum Verlauf der Winkelbeschleunigung im selben Zeitabschnitt, siehe Abbildung \ref{fig:winkelbeschleunigung_py}. Das Maximum der Simulierten Werte ist ca. 50 \% kleiner als die gemessenen Werte. In der Ursache wird die Reibung des Getriebes vermutet. Der gemessene Drehmomentverlauf im ersten Gelenk weißt keinen negativen Anteil auf. Es wird angenommen, dass die Getriebemomente aus messtechnisch erfassten Motorströmen mithilfe der Drehmomentkonstanten und der Getriebeübersetzungsfaktoren berechnet werden. Unter der Annahme, dass die Bremsenergie nicht zurückgewonnen wird, sondern über Bremswiderstände dissipiert, entsprechen die gemessenen Werte für $t>0,8~\text{s}$ dem  Abbremsvorgang. Belegt wird dies mit dem übereinstimmenden Verlauf der Winkelgeschwindigkeit, siehe Abbildung \ref{fig:winkelgeschwindigkeit_py1}. Dass der Drehmomentverlauf dem Geschwindigkeitsverlauf dabei um ca. 0,02 s nachläuft, wird den Energie speichernden Induktivitäten der Antriebe zugeschrieben. Für den Drehmomentverlauf im zweiten Gelenk sind ebenfalls steilere Anstiege infolge eines höheren Rucks des realen Roboters festzustellen. Im Verlauf sind die Maxima der simulierten Drehmomente  ca. 2000 Nm größer als die gemessenen Werte. Die Differenz wird dem nur näherungsweise modellierten Gewichtsausgleich zugeschrieben. Aufgrund der kinematischen Kopplung wirkt der Einfluss des Gewichtsausgleichs am realen Roboter auch indirekt auf das dritte Gelenk womit der Offset zu Beginn der Bewegung erklärt wird. Dass die Größenordnung des Drehmoments im dritten Gelenk nicht erreicht wird, ist der Vernachlässigung der Stator-Massen der Motoren vier und fünf, der Rotor-Trägheit des dritten Motors, sowie und der Steifigkeit, Masse und insbesondere der Massenträgheit des Schlauchpakets zugeschrieben. Mit einem Gewicht von 65,3 kg  ohne elektrische Leitungen (gemäß CAD-Modell) steht die Masse des Schlauchpakets mindestens im Verhältnis 1:5 zur Masse des dritten Verbindungsglieds. Daraus folgt eine zusätzliche Massenträgheit, sowie die Verschiebung des Masseschwerpunkts, unter der Annahme, dass die Hardware ausschließlich am dritten Verbindungsglied befestigt ist. Aufgrund fehlender Daten entfällt dieser Berücksichtigung. Des Weiteren wird für das Drehmoment im dritten Gelenk und alle nachfolgenden die Vernachlässigung  der Reibungseffekte als Ursache für die Abweichung gegenüber den Messdaten vermutet. 
%
\subsection{Auswertung der mechanischen Leistung}
Die simulierten Leistungsdaten des ersten Gelenks stimmen in etwa mit den Werten überein, die auf der Grundlage der RSI Daten berechnet wurden. Der simulierte Verlauf der mechanischen Leistung für das zweite Gelenk weist einen ähnlichen Verlauf wie die aus den realen Daten berechneten Werte auf. Der Nulldurchgang der gemessenen Leistungskurve ist um 1,3 Sekunden nach rechts verschoben.  Dies lässt sich unter Berücksichtigung der RSI-Daten zur Winkelbeschleunigung in Abbildung \ref{fig:winkelbeschleunigung_py} auf die kürzere Abbremsdauer in Bezug auf die im Modell implementierte Bahnplanung  zurückführen. In der gezeigten Bewegung wird die mechanische Leistung in den anderen Antrieben nicht durch das Modell abgebildet. Infolgedessen liegt der Schwerpunkt der Optimierung auf der Anpassung der Roboterkonfiguration zugunsten des Drehmoments im zweiten Gelenk. Aufgrund des hohen Anteils der mechanischen Leistung für das zweite Gelenk im Vergleich zu den anderen Gelenken wird das größte Optimierungspotenzial im Modell berücksichtigt. 
%
\begin{figure}[tbph]
	\centering
	\includegraphics[width=1\linewidth]{images/pmat}
	\caption{mechanische Leistung simuliert}
	\label{fig:pmat}
\end{figure}
%
\begin{figure}[tbph]
	\centering
	\includegraphics[width=1\linewidth]{images/p}
	\caption{mechanische Leistung gemessen}
	\label{fig:p}
\end{figure}
%
\subsection{Schlussfolgerung}
Es wurde die Qualität eines Modells zur Simulation der mechanischen Leistungsaufnahme des Roboters für eine Bewegung mit der Start- und Zielkonfiguration gemäß Tabelle \ref{tab:simu} untersucht. Die ausgewählte Bahn entspricht dem Verfahrweg des Roboters vom letzten Prozesspunkt zur Grundstellung im Fertigungsprogramm $Kleben-Seitenwand$. Für die untersuchte Bewegung erfüllt das Modell den Zweck einer Gütefunktion zur Optimierung der Bewegungsbahn mit dem Ziel der Bewegungsenergieeinsparung. 
%