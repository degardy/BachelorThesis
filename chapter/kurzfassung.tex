\chapter*{Kurzfassung} %*-Variante sorgt dafür, das Abstract nicht im Inhaltsverzeichnis auftaucht
Ziel der vorliegenden Arbeit ist die Untersuchung der Energieeinsparung eines Industrieroboters vom Typ KR210 R2700-2 durch die Umsetzung einer Bahnoptimierung. Dazu wird die Forschungsfrage gestellt, ob ein für die Produktion entwickeltes Roboterprogramm durch Hinzufügen und Verschieben von Via-Punkten ohne signifikante Erhöhung der Bewegungsdauer energetisch optimiert werden kann. Die numerische Optimierung des Via-Punktes erfolgt auf Basis eines simulierten mechanischen Modells. Zur Beantwortung der Forschungsfrage wird eine Abstellbewegung des Roboters vom letzten Prozesspunkt eines Produktionsprogramms in die Grundstellung untersucht. Dieser Verfahrweg wird als repräsentativ für eine Vielzahl von Bewegungsabläufen für Industrieroboter in der Automobilproduktion angesehen, da sie in jedem Takt beim Einfahren einer Karosserie zur Kollisionsvermeidung durchgeführt wird. Nach Validierung des simulierten Modells sowie der Optimierungsergebnisse am realen System wird  eine Energieeinsparung für den beschriebenen Bewegungsablauf von 11,7~\% gegenüber der Initialbewegung erzielt. Basierend auf den Ergebnissen wird empfohlen, eine Potenzialanalyse durchzuführen, in der die Skalierbarkeit des Ansatzes entsprechend der Anzahl baugleicher  Industrieroboter ermittelt wird. Bei positiver Bewertung ist der Versuchsumfang zu erweitern.
\cleardoublepage
