\chapter{Grundlagen}
\label{cha:Grundlagen}

% todo KSYS		

\section{Kinematik-Typen}
	% Spong S. 13 ff
\section{Koordinatentransformation in 3D}
	% Rieber Skript 2
\section{Euklidische Gruppe $SE(3)$}
	%Drehgruppe SO(3)
	 \begin{align*} SE(3) = \left\{{\bm{T}} \ | \ \bm{T}=\begin{bmatrix} {\bm{R}} &\quad {\bm{o}}\\ {0}_{1x3} &\quad 1 \end{bmatrix}\right\}, \ \bm{R} \in {R}^{3x3}, \ {r} \in {R}^{3}  \end{align*}
	%\section{Starre Körper}



	
\section{Newtonsche Axiome}

	 
%\section{Zwangsbedingungen}
%	- %todo holonom
%	Holonome Zwangsbedingungen sind als Gleichungen formulierbar
%	drei Positionskoordinaten definieren die Lage des Massenschwerpunkts eines Arms
%	drei Eulerwinkel definieren die Orientierung des Körpers
%	- %todo nicht holomom
%	Nichtholonome Zwansbedingungen sind nicht als Gleichungen formulierbar, z. B. Ungleichungen
%\section{Hamilton-Prinzip}
%\section{D’Alembertsches Prinzip}


%\section{Kreiselmoment}

\section{Trajektorien-Definition}
Begriffsunterscheidung Pfad/Trajektorie







































