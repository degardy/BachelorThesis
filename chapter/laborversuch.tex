\chapter{Laborversuch}
\section{Messaufbau und Steuerungsarchitektur}
\section{Validierung des Modells}

Ziel der  mechanischen Modellbildung ist die Simulation des Roboter-Energieverbrauchs entlang einer Bewegungsbahn. Aufgrund der Vernachlässigung von Reibungseinflüssen, dem Gewichtsausgleich am  zweiten Gelenk, der Masse und Steifigkeit des Schlauchpakets, sowie dem Drehmoment-Anteil durch die Masse und Masseträgheit der Elektromotoren  wird die Frage formuliert, ob das mechanische Modell der Anforderung gerecht wird, die Roboterdynamik geeignet zu simulieren. Die Ausprägung geeignet liegt vor, wenn Verlauf der simulierten Drehmomente qualitativ reale Effekte nachbildet z. B.  Änderungen in der Geschwindigkeit und Beschleunigung. Daneben wird gefordert, dass die Größenordnung der Drehmomente in den sechs  Gelenken in der Simulation dasselbe Verhältnis abbildet, wie das reale System.  
% Versuchsbeschreibung  	ohne Werkzeug/ Endeffektor 	Geschwindigkeit  	Beschleinigung 	Temperatur kalt und warm 	Messeinrichtung

%	Können Modellparameter nachträglich justiert werden, um das Modell auf andere IR anzuwendenden?
%	Glaubwürddigkeits/Plausibilitätsprüfung des Modells bei Simulation von Roboterbewegungen (Winkelgeschwindigkeit, Endeffektorgeschwingigkeit)
\section{Versuchsdurchführung Optimierung}
Die Zielsetzung definiert die Optimierungsergebnisse aus der Simulation am realen System zu validieren. 
%Versuchsbeschreibung
\section{Auswertung der Optimierungs- und Messergebnisse}