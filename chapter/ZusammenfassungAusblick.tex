\chapter{Zusammenfassung und Ausblick}
Die Einleitung Kapitel \ref{cha:Einleitung} ordnet die Arbeit in den betrieblichen Kontext ein. Es ist hervorzuheben, dass die Steigerung der Energieeffizienz von Industrierobotern einen Anteil zur Erreichung des produktions- und planungsrelevanten Umweltziels \glqq Energieeinsparung\grqq leistet. Aus der Literaturanalyse ergibt sich für die Arbeit im wissenschaftlich-technischen Kontext die Rolle einer praktischen Umsetzung bereits erforschter Ansätze zur Energieeinsparung durch Via-Punkt basierte Bahnoptimierung.

Das Kapitel \ref{cha:Modellbildung} Mechanische Modellbildung beschreibt die Vorwärtskinematik gemäß der Denavit-Hartenberg Konvention und beschreibt die Implementierung des rekursiven Newton-Euler Algorithmus zur Berechnung der im Gelenk auftretenden Getriebemomente entlang einer Bewegungsbahn. Basierend auf der Implementierung des Modells in Matlab kann der Energieverbrauch des Roboters simuliert werden, ohne dass das reale System angesteuert werden muss. Mit dem Modell können die Auswirkungen von Geschwindigkeit, Massenträgheit und der Einfluss des Gewichtsausgleichs analysiert werden. Dadurch kann die in der Ausgangssituation beschriebene Lücke des DOE zur Identifikation der Faktoren für einen energetisch optimierten Via-Punkt geschlossen werden. Das Kapitel zur Modellbildung schließt mit einer Beschreibung der Annahmen und Vernachlässigungen bezüglich des dynamischen Verhaltens. Es wird ausdrücklich angemerkt, dass nur die mechanische Leistung des Roboters beschrieben ist und auf eine Modellierung von Reibungseinflüssen verzichtet wurde. In der Literaturübersicht wird auf Veröffentlichungen verwiesen, die diesen Bereich abdecken.

Im Kapitel \ref{cha:trajektorie} wird die Trajektorie-Planung für die Gelenkwinkel des Roboters eingeführt. Hierbei wird ein Polynom sechster Ordnung verwendet. Zur Validierung werden die berechneten Gelenkwinkel, Winkelgeschwindigkeiten und Winkelbeschleunigungen mit den Bewegungsdaten verglichen, die mithilfe einer RSI-Signalaufzeichnung an der Robotersteuerung erfasst werden.

Im Kapitel \ref{cha:modellvalidierung} Validierung des Roboterdynamik-Modells wird zunächst das RSI Technologiepaket zur Aufzeichnung von Messwerten und Bewegungsdaten an der Robotersteuerung vorgestellt. Unter Berücksichtigung der getroffenen Annahmen wird untersucht, ob das implementierte Modell entlang der berechneten Trajektorie die Roboterdynamik hinreichend genau simuliert. Auf Basis dieser Untersuchung wurde eine Anpassung der RNEA-Implementierung vorgenommen, um den Gewichtsausgleichszylinder am zweiten Gelenk zu berücksichtigen. Anhand eines qualitativen und quantitativen Vergleichs der simulierten und gemessenen Drehmomente sowie der berechneten Leistungsaufnahme konnte das Modell für die im Rahmen der Optimierung untersuchte Bewegungsbahn des Produktionsprogramms  Kleben-Seitenwand validiert werden.

Kapitel \ref{cha:Bahnoptimierung} setzt die Optimierung der Bewegungsbahn um. Hierfür werden zunächst das Optimierungsproblem definiert, eine geeignete Zielfunktion aufgestellt und Nebenbedingungen festgelegt. Zudem werden der numerische Solver und der verwendete Algorithmus beschrieben. Abschließend werden die Ergebnisse der Optimierung analysiert. Es konnte eine Reduktion des Energieverbrauchs für die untersuchte Bewegungsbahn um etwa 7~\% erzielt werden. Des Weiteren wird gezeigt, dass der Optimierer für eine Verwendung, die über den Rahmen der Bachelorarbeit hinausgeht, die Definition zusätzlicher Nebenbedingungen benötigt, um sicherzustellen, dass die kinematischen und dynamischen Grenzen des Roboters in der Optimierung berücksichtigt werden.

In Kapitel \ref{cha:Validierung} Validierung der Optimierungsergebnisse werden die in der Simulation berechneten Energieeinsparungen am realen System mithilfe des optimierten Parametervektors überprüft. Die Ergebnisse der Optimierung konnten im Versuch plausibilisiert werden, wobei eine Energieeinsparung von 11,7~\% erzielt wurde.

Die Arbeit schließt mit einer Bewertung der Ergebnisse in Kapitel \ref{cha:Bewertung} ab. Für die optimierte Bewegungsbahn wird eine Erhöhung der Bewegungsdauer um 0,2 s festgestellt. Diese ist im Kontext einer Produktion unter Berücksichtigung vor- und nachgelagerter Bearbeitungsschritte zu bewerten. Für den isoliert betrachteten Roboter wird die Zunahme der Bewegungsdauer als tolerierbar eingestuft. Die Zielsetzung der Arbeit, einen Via-Punkt modellbasiert zu optimieren, wurde erreicht. Die Forschungsfrage, ob ein für die Produktion entwickeltes Roboterprogramm durch Hinzufügen und Verschieben von Via-Punkten energetisch optimiert werden kann, ohne die Bewegungsdauer signifikant zu erhöhen, konnte bestätigt werden. Die prognostizierte Energieeinsparung von etwa 10~\%, die in der Ausgangssituation mithilfe des DOE ermittelt wurde, konnte sowohl im Modell als auch in einem praktischen Versuch nachgewiesen werden.

In der Arbeit wurde eine Stromwandlung aus Bewegungen mit der Schwerkraft über die Funktion der Antriebe als Generator nicht betrachtet. Im Rahmen eines Literaturüberblicks erfolgt der Verweis auf Forschungsarbeiten, die diese Lücke schließen. Für weitere Untersuchungen zur Optimierung der Roboterbahn sollte zunächst eine Potenzialanalyse durchgeführt werden, in der die Skalierbarkeit des Ansatzes ermittelt wird. Um das eingangs beschriebene Ziel der kontinuierlichen Effizienzsteigerung zu erreichen, ist darüber hinaus die Festlegung der Verantwortlichkeiten für die Umsetzung der Optimierung von Bedeutung. Anzustreben ist die Berücksichtigung der Energieeffizienz neben der Taktzeit bereits bei der Erstprogrammierung der Bewegungsbahn bzw. vor der Inbetriebnahme eines neuen Programmablaufs. Empfehlenswert ist hier eine Anwendung in der Bewegungsabläufe manuell markiert werden, auf deren Basis anschließend eine automatisierte Erweiterung um Via-Punkte und deren Optimierung erfolgt. Ein wichtiges Kriterium ist dabei die Gewährleistung von Kollisionsfreiheit. Wurden im Optimierer bereits dynamische, z. B. kinetische Randbedingungen berücksichtigt, ist im Anschluss an die Optimierung lediglich eine geometrische Überprüfung der Kollisionsfreiheit in einer Simulationssoftware erforderlich. Ein zweiter Ansatz ist die Sicherstellung der Kollisionsfreiheit vor der Optimierung. Vorab könnte ein kartesischer Raum definiert werden, in dem Kollisionsfreiheit vorliegt. Mit dem Ziel, dies möglichst automatisiert ablaufen zu lassen, ist z. B. der Einsatz von Bilderkennungsverfahren zur Bestimmung von EE-Geometrien geeignet, für die keine CAD-Daten vorliegen.




