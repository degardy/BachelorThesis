%\addchap{Anhang A}
%\setcounter{chapter}{1}
%
%\section{Details zu bestimmten theoretischen Grundlagen}
%
%\section{Weitere Details, welche im Hauptteil den Lesefluss behindern}
%
%\addchap{Anhang B}
%\setcounter{chapter}{2}
%\setcounter{section}{0}
%\setcounter{table}{0}
%\setcounter{figure}{0}
%
%\section{Versuchsanordnung}
%
%\section{Liste der verwendeten Messgeräte}
%
%\section{Übersicht der Messergebnisse}
%
%\section{Schaltplan und Bild der Prototypenplatine}

%\clearpage
%
%Diese Seite wurde eingefügt, um zu zeigen, wie sich der Inhalt der Kopfzeile automatisch füllt.
%
%\addchap{Anhang C}
%\setcounter{chapter}{3}
%\setcounter{section}{0}
%\setcounter{table}{0}
%\setcounter{figure}{0}
%
%\section{Struktogramm des Programmentwurfs}
%
%\section{Wichtige Teile des Quellcodes}
%
\addchap{Anhang A}
\setcounter{chapter}{1}
\setcounter{section}{1}
\setcounter{table}{0}
\setcounter{figure}{0}
%
%\section{Einbinden von PDF-Seiten aus anderen Dokumenten}
%
%Auf den folgenden Seiten wird eine Möglichkeit gezeigt, wie aus einem anderen PDF-Dokument komplette Seiten übernommen werden können. Der Nachteil dieser Methode besteht darin, dass sämtliche Formateinstellungen (Kopfzeilen, Seitenzahlen, Ränder, etc.) auf diesen Seiten nicht angezeigt werden. Die Methode wird deshalb eher selten gewählt. Immerhin sorgt das Package \textit{\glqq pdfpages\grqq}~für eine korrekte Seitenzahleinstellung auf den im Anschluss folgenden \glqq nativen\grqq~\LaTeX-Seiten.
%
%Eine bessere Alternative ist, einzelne Seiten mit \textit{\glqq$\backslash$includegraphics\grqq}~einzubinden. Z.B. wenn Inhalte von Datenblättern wiedergegeben werden sollen.

\section{KR210 2700-2 Datenblatt}
\label{add:datenblatt}
\includepdf[pages=1]{C:/Users/denni/Documents/Bachelorarbeit/BachelorThesis/literature/Anhang/DatenblattKR2102700-2.pdf}
\section{DH-Transformation}
\label{add:dh}
%\includepdf[pages=1-2]{C:/Users/denni/Documents/Bachelorarbeit/BachelorThesis/literature/Anhang/dhtrafo_matlab.pdf}
\begin{lstlisting}[language=Matlab]
	function [T,R,R0i] = dhtrafo(q1, q2, q3, q4, q5, q6)
	%Vorwärtskinematik
	% Denavit-Hartenberg-Konvention
	
	theta1 = -q1;
	theta2 = q2+deg2rad(-90);
	theta3 = q3;
	theta4 = q4;
	theta5 = q5;
	theta6 = q6+deg2rad(90);
	
	alpha1 = deg2rad(-90);
	alpha2 = deg2rad(0);
	alpha3 = deg2rad(90);
	alpha4 = deg2rad(-90);
	alpha5 = deg2rad(-90);
	alpha6 = deg2rad(0);
	
	a1 = 330/1000;
	a2 = 1150/1000;
	a3 = 115/1000;
	a4 = 0;
	a5 = 0;
	a6 = 0;
	
	d1 = 645/1000;
	d2 = 0;
	d3 = 0;
	d4 = -1220/1000;
	d5 = 0;
	d6 = 215/1000;
	
	T01 = [ cos(theta1), -sin(theta1)*cos(alpha1), sin(theta1)*sin(alpha1), a1*cos(theta1);
			sin(theta1), cos(theta1)*cos(alpha1), -cos(theta1)*sin(alpha1), a1*sin(theta1);
			0, sin(alpha1), cos(alpha1), d1;
			0, 0, 0, 1];
	
	
	T12 = [ cos(theta2), -sin(theta2)*cos(alpha2), sin(theta2)*sin(alpha2), a2*cos(theta2);
			sin(theta2), cos(theta2)*cos(alpha2), -cos(theta2)*sin(alpha2), a2*sin(theta2);
			0, sin(alpha2), cos(alpha2), d2;
			0, 0, 0, 1];
	
	T23 = [ cos(theta3), -sin(theta3)*cos(alpha3), sin(theta3)*sin(alpha3), a3*cos(theta3);
			sin(theta3), cos(theta3)*cos(alpha3), -cos(theta3)*sin(alpha3), a3*sin(theta3);
			0, sin(alpha3), cos(alpha3), d3;
			0, 0, 0, 1];
	
	T34 = [ cos(theta4), -sin(theta4)*cos(alpha4), sin(theta4)*sin(alpha4), a4*cos(theta4);
			sin(theta4), cos(theta4)*cos(alpha4), -cos(theta4)*sin(alpha4), a4*sin(theta4);
			0, sin(alpha4), 
			 cos(alpha4), d4;
			0, 0, 0, 1];
	
	T45 = [ cos(theta5), -sin(theta5)*cos(alpha5), sin(theta5)*sin(alpha5), a5*cos(theta5);
			sin(theta5), cos(theta5)*cos(alpha5), -cos(theta5)*sin(alpha5), a5*sin(theta5);
			0, sin(alpha5), cos(alpha5), d5;
			0, 0, 0, 1];
	
	T56 = [ cos(theta6), -sin(theta6)*cos(alpha6), sin(theta6)*sin(alpha6), a6*cos(theta6);
			sin(theta6), cos(theta6)*cos(alpha6), -cos(theta6)*sin(alpha6), a6*sin(theta6);
			0, sin(alpha6), cos(alpha6), d6;
			0, 0, 0, 1];
	
	T = cat(3, T01, T12, T23, T34, T45, T56);
	
	R01 = T01(1:3,1:3);
	R12 = T12(1:3,1:3);
	R23 = T23(1:3,1:3);
	R34 = T34(1:3,1:3);
	R45 = T45(1:3,1:3);
	R56 = T56(1:3,1:3);
	
	T02 = T01*T12;
	T03 = T02*T23;
	T04 = T03*T34;
	T05 = T04*T45;
	T06 = T05*T56;
	
	R02 = T02(1:3,1:3);
	R03 = T03(1:3,1:3);
	R04 = T04(1:3,1:3);
	R05 = T05(1:3,1:3);
	R06 = T06(1:3,1:3);
	
	R = cat(3, R01, R12, R23, R34, R45, R56);
	R0i = cat(3, R01, R02, R03, R04, R05, R06);
	
	end
\end{lstlisting}
\label{add:systemparameter}
%
\section{Systemparameter}
Die Massen $m_i$ der Starrkörper $i$ sind dem CAD-Modell des Herstellers entnommen. Hierbei wird angenommen, dass der Werkstoff korrekt zugeordnet ist. 
%
\setlist{noitemsep}
\begin{itemize}
	\item $m_1$ = 535 kg
	\item $m_2$ = 696,3 kg
	\item $m_3$ = 361,6 kg
	\item $m_4$ = 39,676 kg
	\item $m_5$ = 53,619 kg
	\item $m_6$ = 4,528 kg
\end{itemize}
%
Die Trägheitstensoren, werden vom CAD-Modell, bezogen auf das KS$\left\{0\right\}$ vorgegeben. Gleichung \ref{eqn:similarity} zeigt, wie diese einmalig über eine Ähnlichkeitstransformation auf das Körperfeste Koordinatensystem KS$\left\{i\right\}$ transformiert werden. Nachfolgend sind die Trägheitstensoren bezogen auf das KS$\left\{0\right\}$ angegeben. 
%
\setlist{noitemsep}
\begin{itemize}
	\item $I_{1xx} = 17,298~\frac{\text{kg}}{\text{m}^2}$
	\item $I_{1xy} = -2,711~\frac{\text{kg}}{\text{m}^2}$
	\item $I_{1xz} = 1,672~\frac{\text{kg}}{\text{m}^2}$
	\item $I_{1yy} = 32,671~\frac{\text{kg}}{\text{m}^2}$
	\item $I_{1yz} = -0,493~\frac{\text{kg}}{\text{m}^2}$
	\item $I_{1zz} = 29,447~\frac{\text{kg}}{\text{m}^2}$
	\\
	\item $I_{2xx} = 138,135~\frac{\text{kg}}{\text{m}^2}$
	\item $I_{2xy} = -0,012~\frac{\text{kg}}{\text{m}^2}$
	\item $I_{2xz} = 0,387~\frac{\text{kg}}{\text{m}^2}$
	\item $I_{2yy} = 136,664~\frac{\text{kg}}{\text{m}^2}$
	\item $I_{2yz} = 9,231~\frac{\text{kg}}{\text{m}^2}$
	\item $I_{2zz} = 11,838~\frac{\text{kg}}{\text{m}^2}$
	\\
	\item $I_{3xx} = 3,031~\frac{\text{kg}}{\text{m}^2}$
	\item $I_{3xy} = -1,386~\frac{\text{kg}}{\text{m}^2}$
	\item $I_{3xz} = -0,045~\frac{\text{kg}}{\text{m}^2}$
	\item $I_{3yy} = 30,03~\frac{\text{kg}}{\text{m}^2}$
	\item $I_{3yz} = -0,21~\frac{\text{kg}}{\text{m}^2}$
	\item $I_{3zz} = 29,693~\frac{\text{kg}}{\text{m}^2}$
	\\
	\item $I_{4xx} = 0,099~\frac{\text{kg}}{\text{m}^2}$
	\item $I_{4xy} = -0,032~\frac{\text{kg}}{\text{m}^2}$
	\item $I_{4xz} = -0,001~\frac{\text{kg}}{\text{m}^2}$
	\item $I_{4yy} = 0,701~\frac{\text{kg}}{\text{m}^2}$
	\item $I_{4yz} = -9,367\cdot10^{-6}~\frac{\text{kg}}{\text{m}^2}$
	\item $I_{4zz} = 0,698~\frac{\text{kg}}{\text{m}^2}$
	\\
	\item $I_{5xx} = 0,481~\frac{\text{kg}}{\text{m}^2}$
	\item $I_{5xy} = 0,118~\frac{\text{kg}}{\text{m}^2}$
	\item $I_{5xz} = 0.00~\frac{\text{kg}}{\text{m}^2}$
	\item $I_{5yy} = 0,424~\frac{\text{kg}}{\text{m}^2}$
	\item $I_{5yz} = 0,00~\frac{\text{kg}}{\text{m}^2}$
	\item $I_{5zz} = 0,675~\frac{\text{kg}}{\text{m}^2}$
	\\
	\item $I_{6xx} = 0,011~\frac{\text{kg}}{\text{m}^2}$
	\item $I_{6xy} = 5,143\cdot10^{-7}~\frac{\text{kg}}{\text{m}^2}$
	\item $I_{6xz} = -2,003\cdot10^{-6}~\frac{\text{kg}}{\text{m}^2}$
	\item $I_{6yy} = 0,006~\frac{\text{kg}}{\text{m}^2}$
	\item $I_{6yz} = 1,220\cdot10^{07}~\frac{\text{kg}}{\text{m}^2}$
	\item $I_{6zz} = 0,006~\frac{\text{kg}}{\text{m}^2}$
\end{itemize}
%
Nachfolgend ist die Lage der Massenschwerpunkte $C_i^0$ im KS$\left\{0\right\}$ angegeben. 
% 
\begin{itemize}
	\item $r_{0,C_1}^0$ = $\left[-30,103,~441,~452,213\right]$ mm
	\item $r_{0,C_2}^0$ = $\left[330,979,~-222,585,~1094,239\right]$ mm
	\item $r_{0,C_3}^0$ = $\left[705,894,~-8,126,~1908,436\right]$ mm
	\item $r_{0,C_4}^0$ = $\left[1383,185,~4,492,~1909,983\right]$ mm
	\item $r_{0,C_5}^0$ = $\left[1601,767,~33,833,~1909,955\right]$ mm
	\item $r_{0,C_6}^0$ = $\left[1743,222,~-0,007,~1910 051\right]$ mm
\end{itemize}
%
Die Getriebeübersetzung ist der Datei Machine Data (MADA) der Robotersteuerung entnommen.
%
\begin{itemize}
	\item $i_1 = -\frac{7}{1798}$
	\item $i_2 = -\frac{17}{4576}$
	\item $i_3 = \frac{3}{754}$
	\item $i_4 = -\frac{55}{10387}$
	\item $i_5 = -\frac{483}{91834}$
	\item $i_6 = \frac{49400}{6485103}$
\end{itemize}
 %
Nachfolgend ist die Umsetzung der Parameter Transformation via MATLAB\textsuperscript{\textregistered} gezeigt. 
%
%\includepdf[pages=1-4]{C:/Users/denni/Documents/Bachelorarbeit/BachelorThesis/literature/Anhang/parameter_matlab.pdf}
\subsection*{Funktionalitäten}
%
\begin{itemize}
	\setlength{\itemsep}{-1ex}
	\item Vorwärtskinematik für die Ausgangsstellung
	\item Trägheitsmoment $I^{0}_i$ am Massezentrum in kg mm\^{}2
	\item Lage der Massenschwerpunkte im Intertialkoordinatensystem KS\{0\}
	\item $r^{i}_{i,C_i}$
	\item $r^{i}_{i-1,i}$
	\item Masse $m_i$
	\item Getriebe-Übersetzung $i_i$
\end{itemize}
\begin{lstlisting}
	function [I,rec,rae,m,G] = parameter()
	% Berechnung konstante Parameter
\end{lstlisting}
%
\subsection*{Vorwärtskinematik für die Ausgangsstellung}
%
\begin{lstlisting}
	[T,~,~] = dhtrafo(0, 0, 0, 0, 0, 0);
	
	T01 = T(:,:,1);
	T12 = T(:,:,2);
	T23 = T(:,:,3);
	T34 = T(:,:,4);
	T45 = T(:,:,5);
	T56 = T(:,:,6);
	T06 = T01*T12*T23*T34*T45*T56;
	T05 = T01*T12*T23*T34*T45;
	T04 = T01*T12*T23*T34;
	T03 = T01*T12*T23;
	T02 = T01*T12;
	
	R01 = T01(1:3,1:3);
	R02 = T02(1:3,1:3);
	R03 = T03(1:3,1:3);
	R04 = T04(1:3,1:3);
	R05 = T05(1:3,1:3);
	R06 = T06(1:3,1:3);
\end{lstlisting}
%
\subsection*{Trägheitsmoment $I^{0}_i$ am Massezentrum in kg mm\^{}2}
%
\begin{lstlisting}
	I1xx = 17.298;
	I1xy = -2.711;
	I1xz = 1.672;
	I1yx = -2.711;
	I1yy = 32.671;
	I1yz = -0.493;
	I1zx = 1.672;
	I1zy = -0.493;
	I1zz = 29.447;
	
	I2xx = 138.135;
	I2xy = -0.012;
	I2xz = 0.387;
	I2yx = -0.012;
	I2yy = 136.664;
	I2yz = 9.231;
	I2zx = 0.387;
	I2zy = 9.231;
	I2zz = 11.838;
	
	I3xx = 3.031;
	I3xy = -1.386;
	I3xz = -0.045;
	I3yx = -1.386;
	I3yy = 30.03;
	I3yz = -0.21;
	I3zx = -0.045;
	I3zy = -0.21;
	I3zz = 29.693;
	
	I4xx = 0.099;
	I4xy = -0.032;
	I4xz = -0.001;
	I4yx = -0.032;
	I4yy = 0.701;
	I4yz = -9.367E-06;
	I4zx = -0.001;
	I4zy = -9.367E-06;
	I4zz = 0.698;
	
	I5xx = 0.481;
	I5xy = 0.118;
	I5xz = 0.00;
	I5yx = 0.118;
	I5yy = 0.424;
	I5yz = 0.00;
	I5zx = 0.00;
	I5zy = 0.00;
	I5zz = 0.675;
	
	I6xx = 0.011;
	I6xy = 5.143E-07;
	I6xz = -2.003E-06;
	I6yx = 5.143E-07;
	I6yy = 0.006;
	I6yz = 1.220E-07;
	I6zx = -2.003E-06;
	I6zy = 1.220E-07;
	I6zz = 0.006;
	
	% Ähnlichkeitstransformation zur Umrechung in die Körperfesten
	% Koordinatensysteme
	
	I1base = [ I1xx, I1xy, I1xz;
					I1yx, I1yy, I1yz;
					I1zx, I1zy, I1zz];
	
	I1 = R01'*I1base*R01;
	
	I2base = [ I2xx, I2xy, I2xz;
					I2yx, I2yy, I2yz;
					I2zx, I2zy, I2zz];
	
	I2 = R02'*I2base*R02;
	
	I3base = [ I3xx, I3xy, I3xz;
					I3yx, I3yy, I3yz;
					I3zx, I3zy, I3zz];
	
	I3 = R03'*I3base*R03;
	
	I4base = [ I4xx, I4xy, I4xz;
					I4yx, I4yy, I4yz;
					I4zx, I4zy, I4zz];
					
	I4 = R04'*I4base*R04;
	
	I5base = [ I5xx, I5xy, I5xz;
					I5yx, I5yy, I5yz;
					I5zx, I5zy, I5zz];
	
	I5 = R05'*I5base*R05;
	
	I6base = [ I6xx, I6xy, I6xz;
					I6yx, I6yy, I6yz;
					I6zx, I6zy, I6zz];
	
	I6 = R06'*I6base*R06;
	
	I = cat(3,I1,I2,I3,I4,I5,I6);
\end{lstlisting}
%
\subsection*{Lage der Massenschwerpunkte im Intertialkoordinatensystem KS\{0\}}
%
\begin{lstlisting}
	com1 = [-30.103; 4.41; 452.213]/1000;
	com2 = [330.979; -222.585; 1094.239]/1000;
	com3 = [705.894; -8.126; 1908.436]/1000;
	com4 = [1383.185; 4.492; 1909.983]/1000;
	com5 = [1601.767; 33.833; 1909.955]/1000;
	com6 = [1743.222; -0.007; 1910.051]/1000;
\end{lstlisting}
%
\subsection*{$r^{i}_{i,C_i}$}
%
\begin{lstlisting}
	r1e_c1 = (T01)\[com1;1]; r1e_c1 = r1e_c1(1:3,1);
	r2e_c2 = (T02)\[com2;1]; r2e_c2 = r2e_c2(1:3,1);
	r3e_c3 = (T03)\[com3;1]; r3e_c3 = r3e_c3(1:3,1);
	r4e_c4 = (T04)\[com4;1]; r4e_c4 = r4e_c4(1:3,1);
	r5e_c5 = (T05)\[com5;1]; r5e_c5 = r5e_c5(1:3,1);
	r6e_c6 = (T06)\[com6;1]; r6e_c6 = r6e_c6(1:3,1);
	
	rec= cat(3, r1e_c1, r2e_c2, r3e_c3, r4e_c4, r5e_c5, r6e_c6);
\end{lstlisting}
%
\subsection*{$r^{i}_{i-1,i}$}
%
\begin{lstlisting}
	r1a_e = -inv(T01); r1a_e = r1a_e(1:3,4);
	r2a_e = -inv(T12); r2a_e = r2a_e(1:3,4);
	r3a_e = -inv(T23); r3a_e = r3a_e(1:3,4);
	r4a_e = -inv(T34); r4a_e = r4a_e(1:3,4);
	r5a_e = -inv(T45); r5a_e = r5a_e(1:3,4);
	r6a_e = -inv(T56); r6a_e = r6a_e(1:3,4);
	
	rae = cat(3, r1a_e, r2a_e, r3a_e, r4a_e, r5a_e, r6a_e);
\end{lstlisting}
%
\subsection*{Masse $m_i$}
%
\begin{lstlisting}
	m1 = 535;
	m2 = 696.3;
	m3 = 361.6; % +65.3 kg für die Schlauchpaket-Halterung +50 kg für die Antriebe 4 und 5;
	m4 = 39.676;
	m5 = 53.619;
	m6 = 4.528;
	m = [m1,m2,m3,m4,m5,m6];
\end{lstlisting}
%
\subsection*{Getriebe-Übersetzung $i_i$}
%
\begin{lstlisting}
	i1 = -7/1798;
	i2 = -17/4576;
	i3 = 3/754;
	i4 = -55/10387;
	i5 = -483/91834;
	i6 = 49400/6485103;
	G = [i1, i2, i3, i4, i5, i6];
\end{lstlisting}
\begin{lstlisting}
	end
\end{lstlisting}
%
\section{RNEA}
\label{add:rnea}
%\includepdf[pages=1-2]{C:/Users/denni/Documents/Bachelorarbeit/BachelorThesis/literature/Anhang/rnea_matlab.pdf}
\subsection*{Funktionalitäten}
%
\begin{itemize}
	\setlength{\itemsep}{-1ex}
	\item Initialisierung
	\item Vorwärtskinematik
	\item Parameter in der Reihenfolge: $I^{i}_{i}$, $r^{i}_{i,C_i}$, $r^{i}_{i-1,i}$, $m_i$, $i_i$
	\item Winkelgeschwindigkeit $\omega^{i}_i$
	\item Winkelbeschleunigung $\dot\omega^{i}_i$
	\item Lineare Beschleunigung $\ddot p^{i}_i$ (Siciliano) $a_e$ (Spong)
	\item Lineare Beschleunigung $\ddot p^{i}_{C_i}$ (Siciliano) $a_c$ (Spong)
	\item Gewichtsausgleich
	\item Kraft $f^{i}_i$
	\item Drehmoment $\mu^{i}_i$
	\item Drehmoment $\tau_i$ im KS\{i-1\}
\end{itemize}
\begin{lstlisting}
	function [tau,P_mech] = rnea(q1, q2, q3, q4, q5, q6, qd1, qd2, qd3, qd4, qd5, qd6, qdd1, qdd2, qdd3, qdd4, qdd5, qdd6)
\end{lstlisting}
%
\subsection*{Initialisierung}
%
\begin{lstlisting}
	qd = [qd1, qd2, qd3, qd4, qd5, qd6];
	qdd = [qdd1, qdd2, qdd3, qdd4, qdd5, qdd6];
	omega = zeros(3,1,6);
	omega_d = zeros(3,1,6);
	a = zeros(3,1,6);
	a_com = zeros(3,1,6);
	f = zeros(3,1,6);
	mu = zeros(3,1,6);
	tau = zeros(1,1,6);
	P_mech = zeros(1,1,6);
\end{lstlisting}
%
\subsection*{Vorwärtskinematik}
%
\begin{lstlisting}
	[~,R,R0i] = dhtrafo(q1, q2, q3, q4, q5, q6);
\end{lstlisting}
%
\subsection*{Parameter in der Reihenfolge: $I^{i}_{i}$, $r^{i}_{i,C_i}$, $r^{i}_{i-1,i}$, $m_i$, $i_i$}
%
\begin{lstlisting}
	[I,rec,rae,m,G] = parameter();
\end{lstlisting}
%
\subsection*{Winkelgeschwindigkeit $\omega^{i}_i$}
%
\begin{lstlisting}
	for i=1:6
	if i == 1
	omega(:,:,i) = R(:,:,i)'*([0;0;0] + qd(i)*[0;0;1]);
	else
	omega(:,:,i) = R(:,:,i)'*(omega(:,:,i-1) + qd(i)*[0;0;1]);
	end
	end
\end{lstlisting}
%
\subsection*{Winkelbeschleunigung $\dot\omega^{i}_i$}
%
\begin{lstlisting}
	for i=1:1:6
	if i == 1
	omega_d(:,:,i) = R(:,:,i)'*([0;0;0] + qdd(i)*[0;0;1]+cross(qd(i)*[0;0;0], [0;0;1]));
	else
	omega_d(:,:,i) = R(:,:,i)'*(omega_d(:,:,i-1) + qdd(i)*[0;0;1]+cross(qd(i)*omega(:,:,i-1), [0;0;1]));
	end
	end
\end{lstlisting}
%
\subsection*{Lineare Beschleunigung $\ddot p^{i}_i$ (Siciliano) $a_e$ (Spong)}
%
\begin{lstlisting}
	for i=1:6
	if i == 1
	a(:,:,i) = R(:,:,i)'* [0;0;-9.81] + cross(omega_d(:,:,i),rae(:,:,i)) + cross(omega(:,:,i),(cross(omega(:,:,i), rae(:,:,i))));
	else
	a(:,:,i) = R(:,:,i)'*a(:,:,i-1) + cross(omega_d(:,:,i),rae(:,:,i)) + cross(omega(:,:,i),(cross(omega(:,:,i), rae(:,:,i))));
	end
	end
\end{lstlisting}
%
\subsection*{Lineare Beschleunigung $\ddot p^{i}_{C_i}$ (Siciliano) $a_c$ (Spong)}
%
\begin{lstlisting}
	for i=1:1:6
	a_com(:,:,i) = a(:,:,i) + cross(omega_d(:,:,i),rec(:,:,i)) + cross(omega(:,:,i),(cross(omega(:,:,i), rec(:,:,i))));
	end
\end{lstlisting}
%
\subsection*{Gewichtsausgleich}
%
\begin{lstlisting}
	F_gravity = zeros(3, 1, 6); % Vektor für die Gewichtskraft
	for i = 1:6
	F_gravity(:,:,i) = m(i) * [0; 0; -9.81]; % Gewichtskraft in Weltkoordinaten
	F_gravity(:,:,i) = R0i(:,:,i)' * F_gravity(:,:,i); % Gewichtskraft ins Körpersystem transformieren
	end
	
	for i = 3:-1:2
	if i == 3
	fgrav(:,:,i) = eye(3)*F_gravity(:,:,i);
	mugrav(:,:,i) = cross(-fgrav(:,:,i),(rae(:,:,i)+rec(:,:,i)));
	else
	fgrav(:,:,i) = R(:,:,i+1)*fgrav(:,:,i+1) + eye(3)*F_gravity(:,:,i);
	mugrav(:,:,i) = cross(-fgrav(:,:,i),(rae(:,:,i)+rec(:,:,i))) + R(:,:,i+1)*mugrav(:,:,i+1) + R(:,:,i+1)*cross(fgrav(:,:,i+1), rec(:,:,i));
	end
	end
\end{lstlisting}
%
\subsection*{Kraft $f^{i}_i$}
%
\begin{lstlisting}
	for i = 6:-1:1
	if i == 6
	f(:,:,i) = eye(3)*[0;0;0] + m(i)*a_com(:,:,i);
	elseif i == 2
	f(:,:,i) = R(:,:,i+1)*f(:,:,i+1) + m(i)*a_com(:,:,i);
	else
	f(:,:,i) = R(:,:,i+1)*f(:,:,i+1) + m(i)*a_com(:,:,i);
	end
	end
\end{lstlisting}
%
\subsection*{Drehmoment $\mu^{i}_i$}
%
\begin{par}
	Übertragung des Drehmoments von der zweiten auf die ersten Achse ist zu Null gesetzt
\end{par} \vspace{1em}
\begin{lstlisting}
	for i = 6:-1:1
	if i == 6
	mu(:,:,i) = cross(-f(:,:,i),(rae(:,:,i)+rec(:,:,i))) + eye(3)*[0;0;0] + eye(3)*cross([0;0;0], rec(:,:,i)) + I(:,:,i)*omega_d(:,:,i) + cross(omega(:,:,i),(I(:,:,i)*omega(:,:,i)));
	elseif i == 2
	mu(:,:,i) = cross(-f(:,:,i),(rae(:,:,i)+rec(:,:,i))) + R(:,:,i+1)*mu(:,:,i+1) + R(:,:,i+1)*cross(f(:,:,i+1), rec(:,:,i)) + I(:,:,i)*omega_d(:,:,i) + cross(omega(:,:,i),(I(:,:,i)*omega(:,:,i)))- mugrav(:,:,i);
	elseif i == 1
	mu(:,:,i) = cross(-f(:,:,i),(rae(:,:,i)+rec(:,:,i))) + R(:,:,i+1)*cross(f(:,:,i+1), rec(:,:,i)) + I(:,:,i)*omega_d(:,:,i) + cross(omega(:,:,i),(I(:,:,i)*omega(:,:,i)));
	else
	mu(:,:,i) = cross(-f(:,:,i),(rae(:,:,i)+rec(:,:,i))) + R(:,:,i+1)*mu(:,:,i+1) + R(:,:,i+1)*cross(f(:,:,i+1), rec(:,:,i)) + I(:,:,i)*omega_d(:,:,i) + cross(omega(:,:,i),(I(:,:,i)*omega(:,:,i)));
	end
	end
\end{lstlisting}
%
\subsection*{Drehmoment $\tau_i$ im KS\{i-1\}}
%
\begin{lstlisting}
	for i = 1:1:6
	tau(i) = transpose(mu(:,:,i)) * transpose(R(:,:,i))*[0;0;1];%* G(i)
	end
	for i = 1:1:6
	P_mech(i) = tau(i)*qd(i);%/G(i);
	end
\end{lstlisting}
\begin{lstlisting}
	end
\end{lstlisting}
\section{Bahnplanung}
\label{add:traj}
%\includepdf[pages=1]{C:/Users/denni/Documents/Bachelorarbeit/BachelorThesis/literature/Anhang/trajectorie_planning_sixth_order_matlab.pdf}
\subsection*{Funktionalitäten}
%
\begin{itemize}
	\setlength{\itemsep}{-1ex}
	\item Initialisierung
	\item Aufstellen der Polynome
	\item Definition der Nebenbedingungen
	\item Berechnnung der Koeffizienten $a_0,...,a_6$
	\item Berechnung der Trajektorie
\end{itemize}
\begin{lstlisting}
	function [q,qd,qdd,t] = trajectorie_planning_sixth_order(ts, te, tv, stepsize, qs, qds, qdds, qv, qe, qde, qdde)
\end{lstlisting}
\begin{lstlisting}
	% Bahnplanung über ein Polynom 6-ter Ordnung inkl. Gelenkwinkel-Definition
	% des Via-Punkts
\end{lstlisting}
%
\subsection*{Initialisierung}
%
\begin{lstlisting}
	t = ts:stepsize:te;
	q = zeros(1,1)*(length(t));
	qd = zeros(1,1)*(length(t));
	qdd = zeros(1,1)*(length(t));
\end{lstlisting}
%
\subsection*{Aufstellen der Polynome}
%
\begin{lstlisting}
	T = [ 1, ts, ts^2, ts^3, ts^4, ts^5, ts^6;
	0, 1, 2*ts, 3*ts^2, 4*ts^3, 5*ts^4, 6*ts^5;
	0, 0, 2, 6*ts, 12*ts^2, 20*ts^3, 30*ts^4;
	1, tv, tv^2, tv^3, tv^4, tv^5, tv^6;
	1, te, te^2, te^3, te^4, te^5, te^6;
	0, 1, 	 2*te, 3*te^2, 4*te^3, 5*te^4, 6*te^5;
	0, 0, 2, 6*te, 12*te^2, 20*te^3, 30*te^4];
\end{lstlisting}
%
\subsection*{Definition der Nebenbedingungen}
%
\begin{lstlisting}
	p = [qs, qds, qdds, qv, qe, qde, qdde];
\end{lstlisting}
%
\subsection*{Berechnnung der Koeffizienten $a_0,...,a_6$}
%
\begin{par}
	a = [a0, a1, a2, a3, a4, a5, a6]'
\end{par} \vspace{1em}
\begin{lstlisting}
	a = T\p';
\end{lstlisting}
%
\subsection*{Berechnung der Trajektorie}
%
\begin{lstlisting}
	for index = 1:length(t)
	q(index) = a(1,1) + a(2,1)*t(index) + a(3,1)*(t(index))^2 + a(4,1)*(t(index))^3 +a(5,1)*(t(index))^4 + a(6,1)*(t(index))^5 + a(7,1)*(t(index))^6;
	qd(index) = a(2,1) + 2*a(3,1)*(t(index)) + 3*a(4,1)*(t(index))^2 +4*a(5,1)*(t(index))^3 + 5*a(6,1)*(t(index))^4 + 6*a(7,1)*(t(index))^5;
	qdd(index) = 2*a(3,1) + 6*a(4,1)*(t(index)) +12*a(5,1)*(t(index))^2 + 20*a(6,1)*(t(index))^3 + 30*a(7,1)*(t(index))^4;
	end
\end{lstlisting}
\begin{lstlisting}
	end
\end{lstlisting}
%
\setcounter{chapter}{2}
\setcounter{section}{5}
\setcounter{table}{0}
\setcounter{figure}{0}
%
\section{Testsimulation Bahnplanung}
\label{add:sim}
%\includepdf[pages=1-2]{C:/Users/denni/Documents/Bachelorarbeit/BachelorThesis/literature/Anhang/simulation_matlab.pdf}
\subsection*{Funktionalitäten}
%
\begin{itemize}
	\setlength{\itemsep}{-1ex}
	\item Trajektorie Kleben Seitenwand
	\item Bewegung 1 (home -\ensuremath{>} Vorposition)
	\item Bewegung 3 (letzter Punkt der Trajektorie Kleben Seitenwand -\ensuremath{>} home)
	\item Bahnplanung
	\item Modellberechnung der Drehmomente, Leistungsaufnahme
	\item Daten Vorverarbeitung
	\item Anzeige der Daten
	\item Erzeuge die Abbildungen
\end{itemize}
\begin{lstlisting}
	hold off
	close all
	clear
	clc
\end{lstlisting}
%
\subsection*{Trajektorie Kleben Seitenwand}
%
\begin{lstlisting}
	te = 1.2; % Bewegungsdauer
\end{lstlisting}
%
\subsection*{Bewegung 1 (home -\ensuremath{>} Vorposition)}
%
\begin{par}
	\% Startwert Gelenkwinkel qs1 = deg2rad(-7.61); qs2 = deg2rad(-119.27); qs3 = deg2rad(88.49-90); qs4 = deg2rad(10.27); qs5 = deg2rad(32.41); qs6 = deg2rad(-10.19);
\end{par} \vspace{1em}
\begin{par}
	\% Zielwert Gelenkwinkel qe1 = deg2rad(-14.83); qe2 = deg2rad(-105.81); qe3 = deg2rad(136.16-90); qe4 = deg2rad(-27.67); qe5 = deg2rad(-33.44); qe6 = deg2rad(22.89);
\end{par} \vspace{1em}
\begin{lstlisting}
	% optimierte Via-Punkte
	% qvd = [];
	% qv1 = deg2rad(qvd(1));
	% qv2 = deg2rad(qvd(2));
	% qv3 = deg2rad(qvd(3));
	% qv4 = deg2rad(qvd(4));
	% qv5 = (deg2rad(qvd(5))+(qs5+qe5)*1/2)*1/2; % Für den Fall, dass der Wert zu dicht an den Grenzen liegt, was eine hohe Beschleunigung zur Folge hat, wird das Mittel aus dem optimierten und dem initialen Via-Punkt gebildet
	% qv6 = (deg2rad(qvd(6))+(qs6+qe6)*1/2)*1/2;
\end{lstlisting}
%
\subsection*{Bewegung 3 (letzter Punkt der Trajektorie Kleben Seitenwand -\ensuremath{>} home)}
%
\begin{par}
	Startwert Gelenkwinkel
\end{par} \vspace{1em}
\begin{lstlisting}
	qs1 = deg2rad(-53.8);
	qs2 = deg2rad(-70.34);
	qs3 = deg2rad(98.82-90);
	qs4 = deg2rad(-69.87);
	qs5 = deg2rad(-58.7);
	qs6 = deg2rad(55.7);
	
	% Zielwert Gelenkwinkel
	qe1 = deg2rad(-7.61);
	qe2 = deg2rad(-119.27);
	qe3 = deg2rad(88.49-90);
	qe4 = deg2rad(10.27);
	qe5 = deg2rad(32.41);
	qe6 = deg2rad(-10.19);
	
	% % Startwert Via-Punkte
	% qv1 = (qs1+qe1)*1/2;
	% qv2 = (qs2+qe2)*1/2;
	% qv3 = (qs3+qe3)*1/2;
	% qv4 = (qs4+qe4)*1/2;
	% qv5 = (qs5+qe5)*1/2;
	% qv6 = (qs6+qe6)*1/2;
	
	% optimierte Via-Punkte
	qvu = [-25.0007 -96.0711 8.8200 10.2700 -47.7822 -10.1900];
	qv1 = deg2rad(qvu(1));
	qv2 = deg2rad(qvu(2));
	qv3 = deg2rad(qvu(3));
	qv4 = (deg2rad(qvu(4))+(qs4+qe4)*1/2)*1/2;
	qv5 = (deg2rad(qvu(5))+(qs5+qe5)*1/2)*1/2;
	qv6 = (deg2rad(qvu(6))+(qs6+qe6)*1/2)*1/2;
\end{lstlisting}
%
\subsection*{Bahnplanung}
%
\begin{lstlisting}
	stepsize = 0.004;
	ts = 0; % Startzeit
	tv = (te-ts)/2; % Via-Punkt Zeitpunkt
	
	% Start- und Endgeschwindigkeiten
	qds1 = 0; qde1 = 0;
	qds2 = 0; qde2 = 0;
	qds3 = 0; qde3 = 0;
	qds4 = 0; qde4 = 0;
	qds5 = 0; qde5 = 0;
	qds6 = 0; qde6 = 0;
	
	% Start- und Endbeschleunigungen
	qdds1 = 0; qdde1 = 0;
	qdds2 = 0; qdde2 = 0;
	qdds3 = 0; qdde3 = 0;
	qdds4 = 0; qdde4 = 0;
	qdds5 = 0; qdde5 = 0;
	qdds6 = 0; qdde6 = 0;
	
	qs = [qs1,qs2,qs3,qs4,qs5,qs6];
	qe = [qe1,qe2,qe3,qe4,qe5,qe6];
	qv = [qv1,qv2,qv3,qv4,qv5,qv6];
	qds = [qds1,qds2,qds3,qds4,qds5,qds6];
	qdds = [qdds1,qdds2,qdds3,qdds4,qdds5,qdds6];
	qde = [qde1,qde2,qde3,qde4,qde5,qde6];
	qdde = [qdde1,qdde2,qdde3,qdde4,qdde5,qdde6];
	
	% Bahnplanung Polynom 6-Ordnung
	[q1,qd1,qdd1,~] = trajectorie_planning_sixth_order(ts, te, tv, stepsize, qs(1), qds(1), qdds(1), qv(1), qe(1), qde(1), qdde(1));
	[q2,qd2,qdd2,~] = trajectorie_planning_sixth_order(ts, te, tv, stepsize, qs(2), qds(2), qdds(2), qv(2), qe(2), qde(2), qdde(2));
	[q3,qd3,qdd3,~] = trajectorie_planning_sixth_order(ts, te, tv, stepsize, qs(3), qds(3), qdds(3), qv(3), qe(3), qde(3), qdde(3));
	[q4,qd4,qdd4,~] = trajectorie_planning_sixth_order(ts, te, tv, stepsize, qs(4), qds(4), qdds(4), qv(4), qe(4), qde(4), qdde(4));
	[q5,qd5,qdd5,~] = trajectorie_planning_sixth_order(ts, te, tv, stepsize, qs(5), qds(5), qdds(5), qv(5), qe(5), qde(5), qdde(5));
	[q6,qd6,qdd6,t] = trajectorie_planning_sixth_order(ts, te, tv, stepsize, qs(6), qds(6), qdds(6), qv(6), qe(6), qde(6), qdde(6));
\end{lstlisting}
%
\subsection*{Modellberechnung der Drehmomente, Leistungsaufnahme}
%
\begin{lstlisting}
	% Initialisierung
	tau = zeros(length(t));
	Pmech = zeros(length(t));
	Pmech_diss = zeros(length(t));
	Pmech_sink = zeros(length(t));
	
	% RNEA
	for index = 1:1:(length(t))
	[ret_tau,ret_Pmech] = rnea(q1(index), q2(index), q3(index), q4(index), q5(index), q6(index), qd1(index), qd2(index), qd3(index), qd4(index), qd5(index), qd6(index), qdd1(index), qdd2(index), qdd3(index), qdd4(index), qdd5(index), qdd6(index));
	tau(index,1) = ret_tau(1);
	tau(index,2) = ret_tau(2);
	tau(index,3) = ret_tau(3);
	tau(index,4) = ret_tau(4);
	tau(index,5) = ret_tau(5);
	tau(index,6) = ret_tau(6);
	Pmech(index,1) = ret_Pmech(1);
	Pmech(index,2) = ret_Pmech(2);
	Pmech(index,3) = ret_Pmech(3);
	Pmech(index,4) = ret_Pmech(4);
	Pmech(index,5) = ret_Pmech(5);
	Pmech(index,6) = ret_Pmech(6);
	end
\end{lstlisting}
%
\subsection*{Daten Vorverarbeitung}
%
\begin{lstlisting}
	for index = 1:1:(length(t))
	for var = 1:6
	if Pmech(index,var) > 0
	Pmech_sink(index,var) = Pmech(index,var);
	else
	Pmech_diss(index,var) = Pmech(index,var);
	Pmech_sink(index,var) = 0;
	
	end
	end
	end
\end{lstlisting}
%
\subsection*{Anzeige der Daten}
%
\begin{lstlisting}
	disp('qv1');disp(rad2deg(qv1));
	disp('qv2');disp(rad2deg(qv2));
	disp('qv3');disp(rad2deg(qv3));
	disp('qv4');disp(rad2deg(qv4));
	disp('qv5');disp(rad2deg(qv5));
	disp('qv6');disp(rad2deg(qv6));
	torque = sum(sum(abs(tau)))/length(t); disp('Mittelwert der Summe des Betrags der Getriebe-Drehmomente in Nm'); disp(torque);
	P_mech = sum(sum((Pmech_sink))/length(t)); disp('Mittelwert der aufgenommenen mechanischen Leistung in W'); disp(P_mech);
	P_mech_diss = sum(sum((Pmech_diss))/length(t))*(te-ts); disp('Dissipierte Energie in J'); disp(P_mech_diss);
	E_mech = sum(sum(Pmech_sink)/length(t))*(te-ts); disp('Aufgenommene Energie in J'); disp(E_mech);
\end{lstlisting}
%
\subsection*{Erzeuge die Abbildungen}
%
\begin{lstlisting}
	show_graphics(q1, q2, q3, q4, q5, q6, qd1, qd2, qd3, qd4, qd5, qd6, qdd1, qdd2, qdd3, qdd4, qdd5, qdd6, t, tau, Pmech)
\end{lstlisting}
%
%
\section{Berechnung der Zielfunktion}
%
\label{add:zielfunktion}
\subsection*{Funktionalitäten}
%
\begin{itemize}
	\setlength{\itemsep}{-1ex}
	\item Bahnplanung
	\item jacobi
	\item Modellberechnung der Drehmomente, Leistungsaufnahme
	\item Daten Vorverarbeitung
	\item Anzeige der Daten
\end{itemize}
\begin{lstlisting}
	function [E_mech] = calc_objective(qs,qe,qv,te)
\end{lstlisting}
\begin{lstlisting}
	% Berechung der Zielfunktion (Energieverbrauch) über die nicht
	% dissipierte mechanische Leistung
\end{lstlisting}
%
\subsection*{Bahnplanung}
%
\begin{lstlisting}
	stepsize = 0.004;
	ts = 0;
	tv = (te-ts)/2;
	
	qds1 = 0; qde1 = 0;
	qds2 = 0; qde2 = 0;
	qds3 = 0; qde3 = 0;
	qds4 = 0; qde4 = 0;
	qds5 = 0; qde5 = 0;
	qds6 = 0; qde6 = 0;
	
	qdds1 = 0; qdde1 = 0;
	qdds2 = 0; qdde2 = 0;
	qdds3 = 0; qdde3 = 0;
	qdds4 = 0; qdde4 = 0;
	qdds5 = 0; qdde5 = 0;
	qdds6 = 0; qdde6 = 0;
	
	qds = [qds1,qds2,qds3,qds4,qds5,qds6];
	qdds = [qdds1,qdds2,qdds3,qdds4,qdds5,qdds6];
	qde = [qde1,qde2,qde3,qde4,qde5,qde6];
	qdde = [qdde1,qdde2,qdde3,qdde4,qdde5,qdde6];
	
	[q1,qd1,qdd1,~] = trajectorie_planning_sixth_order(ts, te, tv, stepsize, qs(1), qds(1), qdds(1), qv(1), qe(1), qde(1), qdde(1));
	[q2,qd2,qdd2,~] = trajectorie_planning_sixth_order(ts, te, tv, stepsize, qs(2), qds(2), qdds(2), qv(2), qe(2), qde(2), qdde(2));
	[q3,qd3,qdd3,~] = trajectorie_planning_sixth_order(ts, te, tv, stepsize, qs(3), qds(3), qdds(3), qv(3), qe(3), qde(3), qdde(3));
	[q4,qd4,qdd4,~] = trajectorie_planning_sixth_order(ts, te, tv, stepsize, qs(4), qds(4), qdds(4), qv(4), qe(4), qde(4), qdde(4));
	[q5,qd5,qdd5,~] = trajectorie_planning_sixth_order(ts, te, tv, stepsize, qs(5), qds(5), qdds(5), qv(5), qe(5), qde(5), qdde(5));
	[q6,qd6,qdd6,t] = trajectorie_planning_sixth_order(ts, te, tv, stepsize, qs(6), qds(6), qdds(6), qv(6), qe(6), qde(6), qdde(6));
\end{lstlisting}
%
\subsection*{Modellberechnung der Drehmomente, Leistungsaufnahme}
%
\begin{lstlisting}
	% Initialisierung
	tau = zeros(length(t));
	Pmech = zeros(length(t));
	Pmech_diss = zeros(length(t));
	Pmech_sink = zeros(length(t));
	
	for index = 1:1:(length(t))
	[ret_tau,ret_Pmech] = rnea(q1(index), q2(index), q3(index), q4(index), q5(index), q6(index), qd1(index), qd2(index), qd3(index), qd4(index), qd5(index), qd6(index), qdd1(index), qdd2(index), qdd3(index), qdd4(index), qdd5(index), qdd6(index));
	tau(index,1) = ret_tau(1);
	tau(index,2) = ret_tau(2);
	tau(index,3) = ret_tau(3);
	tau(index,4) = ret_tau(4);
	tau(index,5) = ret_tau(5);
	tau(index,6) = ret_tau(6);
	Pmech(index,1) = ret_Pmech(1);
	Pmech(index,2) = ret_Pmech(2);
	Pmech(index,3) = ret_Pmech(3);
	Pmech(index,4) = ret_Pmech(4);
	Pmech(index,5) = ret_Pmech(5);
	Pmech(index,6) = ret_Pmech(6);
	end
\end{lstlisting}
%
\subsection*{Daten Vorverarbeitung}
%
\begin{lstlisting}
	for index = 1:1:(length(t))
	for var = 1:6
	if Pmech(index,var) > 0
	Pmech_sink(index,var) = Pmech(index,var);
	else
	Pmech_diss(index,var) = Pmech(index,var);
	Pmech_sink(index,var) = 0;
	end
	end
	end
\end{lstlisting}
%
\subsection*{Anzeige der Daten}
%
\begin{lstlisting}
	torque = sum(sum((tau).^2))/length(t);
	E_mech = sum(sum((Pmech_sink))/length(t))*(te-ts); disp('E in J'); disp(E_mech);
\end{lstlisting}
\begin{lstlisting}
	end
\end{lstlisting}
%
%
\section{Optimierung}
\label{acc:optimierer}
%
\subsection*{Funktionalitäten}
%
\begin{itemize}
	\setlength{\itemsep}{-1ex}
	\item KlebenSeitenwand
	\item Bewegung 1 (home -\ensuremath{>} Vorposition)
	\item Bewegung 3 (letzter Punkt der Trajektorie Kleben Seitenwand -\ensuremath{>} home)
	\item Startwert Via-Punkte
	\item Initial-Trajektorie-Definition
	\item Definition Optimierer
\end{itemize}
\begin{lstlisting}
	hold off
	close all
	clear
	clc
\end{lstlisting}
%
\subsection*{KlebenSeitenwand}
%
\subsection*{Bewegung 1 (home -\ensuremath{>} Vorposition)}
%
\begin{par}
	\% Startwert Gelenkwinkel qs1 = deg2rad(-7.61); qs2 = deg2rad(-119.27); qs3 = deg2rad(88.49-90); qs4 = deg2rad(10.27); qs5 = deg2rad(32.41); qs6 = deg2rad(-10.19);
\end{par} \vspace{1em}
\begin{par}
	\% Zielwert Gelenkwinkel qe1 = deg2rad(-14.83); qe2 = deg2rad(-105.81); qe3 = deg2rad(136.16-90); qe4 = deg2rad(-27.67); qe5 = deg2rad(-33.44); qe6 = deg2rad(22.89);
\end{par} \vspace{1em}
%
\subsection*{Bewegung 3 (letzter Punkt der Trajektorie Kleben Seitenwand -\ensuremath{>} home)}
%
\begin{par}
	Startwert Gelenkwinkel
\end{par} \vspace{1em}
\begin{lstlisting}
	qs1 = deg2rad(-53.8);
	qs2 = deg2rad(-70.34);
	qs3 = deg2rad(98.82-90);
	qs4 = deg2rad(-69.87);
	qs5 = deg2rad(-58.7);
	qs6 = deg2rad(55.7);
	
	% Zielwert Gelenkwinkel
	qe1 = deg2rad(-7.61);
	qe2 = deg2rad(-119.27);
	qe3 = deg2rad(88.49-90);
	qe4 = deg2rad(10.27);
	qe5 = deg2rad(32.41);
	qe6 = deg2rad(-10.19);
\end{lstlisting}
%
\subsection*{Startwert Via-Punkte}
%
\begin{lstlisting}
	denum = 1; % Testen von Startwerten, die von Mittelpunkt abweichen
	qv1 = (qs1+qe1)*1/2;%-(abs(qs1-qe1)/2)/denum;
	qv2 = (qs2+qe2)*1/2;%-(abs(qs1-qe1)/2)/denum;
	qv3 = (qs3+qe3)*1/2;%-(abs(qs1-qe1)/2)/denum;
	qv4 = (qs4+qe4)*1/2;%-(abs(qs1-qe1)/2)/denum;
	qv5 = (qs5+qe5)*1/2;%-(abs(qs1-qe1)/2)/denum;
	qv6 = (qs6+qe6)*1/2;%-(abs(qs1-qe1)/2)/denum;
\end{lstlisting}
%
\subsection*{Initial-Trajektorie-Definition}
%
\begin{lstlisting}
	te = 1.2;
	qs = [qs1,qs2,qs3,qs4,qs5,qs6];
	qe = [qe1,qe2,qe3,qe4,qe5,qe6];
	qv = [qv1,qv2,qv3,qv4,qv5,qv6];
\end{lstlisting}
%
\subsection*{Definition Optimierer}
%
\begin{par}
	Zielfunktion
\end{par} \vspace{1em}
\begin{lstlisting}
	objective = @(qv) calc_objective(qs,qe,qv,te);
	% Startwert
	x0 = qv;
	% Anzeige des Energieverbrauchs beim Start der Optimierung
	disp(['initial Objective: ' num2str(objective(x0))]);
	% Gleichheitsnebenbedingungen
	A = [];
	b = [];
	% Ungleichheitsnebenbedingungen
	Aeq = [];
	beq = [];
	% Grenzen des Parametervektors
	lb = (qv-(abs(qs-qe))/2);
	ub = (qv+(abs(qs-qe))/2);
	% Nichtlineare Nebenbedingungen
	nonlcon = [];
	% Auswahl des Solvers fmincon
	% Algorithmus sequentielle quadratische Programmierung
	% max. Anzahl der Iterationen = 25
	% und Ausgabe des Zielfunktionswertes mit jeder Iteration
	options = optimoptions(@fmincon,'Algorithm','sqp', 'MaxIterations', 25,'PlotFcn',@optimplotfval);
	% Ausführen der Optimierung
	[x,fval,ef,output,lambda] = fmincon(objective,x0,A,b,Aeq,beq,lb,ub,nonlcon,options);
	% Anzeige des energieoptimalen Parametervektors
	disp(x*180/pi)
\end{lstlisting}
\color{lightgray} \begin{lstlisting}
% Startwert
E in J
2.0013e+03

% Zielwert
E in J
1.8074e+03

% identifzierter Parametervektor
-24.8857 -96.0716 8.8200 10.2700 -47.5666 -10.1900

\end{lstlisting} \color{black}
%
\includegraphics [width=4in]{images/optimization_01}
%
\section{Vergleich der simulierten Verläufe für den initialen und justierten, energieoptimalen Parametervektor}
\label{acc:optupjust}
%
in Abbildung \ref{fig:posoptfinal} wird ersichtlich, dass die Gelenkwinkel $q_{i,justiert}(t)$ innerhalb der Start- und Zielwinkel liegen.
%
\begin{equation}
	q_{i,justiert}(t) \in [q_{i,s};q_{i,e}] ~\forall~ i \in \{1,2,3,4,5,6\}
\end{equation}
%
Die erreichten Winkelgeschwindigkeiten liegen unterhalb der in \ref{add:datenblatt} notierten Winkelgeschwindigkeiten bei Nenn-Traglast.
%
\begin{figure}[tbph]
	\centering
	\includegraphics[width=1\linewidth]{images/Optimierungsergebnisse_up/posoptfinal}
	\caption{Gelenkwinkelverläufe der Initialbahn und justierten energieoptimierten Bewegungsbahn}
	\label{fig:posoptfinal}
\end{figure}
%
\begin{figure}[tbph]
	\centering
	\includegraphics[width=1\linewidth]{images/Optimierungsergebnisse_up/veloptfinal}
	\caption{Winkelgeschwindigkeitsverläufe der Initialbahn und justierten energieoptimierten Bewegungsbahn}
	\label{fig:veloptfinal}
\end{figure}
%
\begin{figure}[tbph]
	\centering
	\includegraphics[width=1\linewidth]{images/Optimierungsergebnisse_up/accoptfinal}
	\caption{Winkelbeschleunigungsverläufe der Initialbahn und justierten energieoptimierten Bewegungsbahn}
	\label{fig:accoptfinal}
\end{figure}
%
Der qualitative Verlauf der aufgenommenen Leistung für den justierten energieoptimalen Parametervektor entspricht dem des energieoptimalen Parametervektors. 
%
\begin{figure}[tbph]
	\centering
	\includegraphics[width=1\linewidth]{images/Optimierungsergebnisse_up/tauoptfinal}
	\caption{Drehmomentverläufe der Initialbahn und justierten energieoptimierten Bewegungsbahn}
	\label{fig:tauoptfinal}
\end{figure}
%
\begin{figure}[tbph]
	\centering
	\includegraphics[width=1\linewidth]{images/Optimierungsergebnisse_up/poptfinal}
	\caption{Verläufe der aufgenommen Leistung je Gelenkwinkel für die Initialbahn und justierte energieoptimierte Bewegungsbahn}
	\label{fig:poptfinal}
\end{figure}
