\chapter{Bahnoptimierung}
\cite{Eggers.2019}
\cite{Engelke.2008}
\cite{Ziaukas.2017}
\cite{Hansen.2012}
\cite{Spong.2020}
\section{Zielsetzung}

Minimierung des mechanischen Energieverbrauchs durch Anpassung der Bewegungsbahn.  

Unterscheidung Prozessbezogener Bewegungen  und dem Verfahren auf Halte- bzw. Vorpositionen z. B. der Fahrt in die Home-Stellung

Explizite Betrachtung der Leistungsaufnahme. Es wird angenommen, das die generatorisch erzeugte Leistung während des Abbremsvorgangs über Bremswiderstände dissipiert. \cite[S.~19]{Eggers.2019}

\section{Optimierungsparameter}
Der Parametervektor $\bm{p}_VP$ \cite[S.~49]{Eggers.2019} umfasst die Gelenkwinkel des nach der Hälfte der Verfahrdauer erreichten Via-Punkts.  

\section{Zielfunktion}
Zur Identifizierung des Optimums werden die Zielfunktionen \ref{eqn:torque} \cite[S-~1]{Hansen.2012} und \ref{eqn:emech} \cite[S.~57]{Eggers.2019} aufgestellt. Die Integrationsgrenzen $t_s$ und $t_e$ entsprechen dem Start- bzw. Zielzeitput. Der Ausdruck \ref{eqn:torque} erzielt durch Quadrieren den Vorteil, dass hohe Drehmomente stärker gewichtet werden.\textbf{}
%
\begin{equation}
	\label{eqn:torque}
	J_{\tau}(\bm{p}_{VP}) = \frac{1}{2}\int_{ts}^{te}\sum_{i=1}^{n}\left|\tau_i(t)\right|^2~dt
\end{equation}
%
Von einem Vorschlag der Zielfunktion \ref{eqn:emechges} \cite[S.~1216]{Saravanan.2008} wird Abstand genommen, da hierbei die generatorisch umgewandelte Leistung während des Abbremsvorgangs als aufgenommene (zugeführte) Leistung der Verbraucher bewertet wird \cite[S.~1]{Hansen.2012}. 
%
\begin{equation}
	\label{eqn:emechges}
	J_{P_{{ges}}}(\bm{p}_{VP}) = \int_{ts}^{te}\sum_{i=1}^{n}\left|\tau_i(t)\dot{q_i}\right|^2~dt
\end{equation}
%
Die Zielfunktion \ref{eqn:emech} entspricht dem physikalischen Ausdruck der verrichteten Arbeit mit positiven Vorzeichen. Infolgedessen wird ausschließlich die aufgenommene (zugeführte) Leistung berücksichtigt. In der Anwendung hat sich diese Funktion am besten bewährt.
%
\begin{equation}
	\label{eqn:emech}
	J_{P_{zu}}
	(\bm{p}_{VP}) 
	= E_{mech_{+}}(\bm{p}_{VP}) 
	=\int_{ts}^{te}P_{zu}(\bm{p}_{VP,t})~dt
\end{equation}
%
Die numerische Implementierung der Zielfunktion in  MATLAB\textsuperscript{\textregistered} entspricht der Gleichung  \ref{eqn:numerischezielfunktion}. 
Die Schrittweite ist in Anlehnung an die Abtastung der Messeinrichtung  mit $\Delta t = 0.004 \text{s}$ festgelegt. Die Anzahl der Schritte $m$  wird gemäß $(t_s-t_e)/\Delta t$ berechnet.
%
\begin{equation}
	\label{eqn:numerischezielfunktion}
	J_{P_{zu}}
	(\bm{p}_{VP}) 
	= E_{mech_{+}}(\bm{p}_{VP}) 
	= \sum_{k=1}^{m} P_{zu}(\bm{p}_{VP,t})~\Delta t
\end{equation}
%



\section{Nebenbedingungen}
Die Verfahrdauer bleibt gegenüber der Initialbahn konstant und wird implizit durch die feste Definition des Zielzeitpunks $t_e$ vorgegeben. 

\section{Solver und Optimierungsalgorithmus}