\chapter*{Abstract} %*-Variante sorgt dafür, das Abstract nicht im Inhaltsverzeichnis auftaucht
This thesis aims to investigate the energy savings of an industrial robot by implementing path optimization. The research question is whether a robot programme developed for car production can be optimised in terms of energy by adding and shifting via points without significantly increasing the movement duration. The numerical optimization of the via point is based on a simulated mechanical model. Therefore, the movement of the robot from the last process point of a production programme to the home position is investigated. This path is considered representative of industrial robots' motion in automotive production since it is performed in every cycle to avoid collisions. After validation of the simulated model as well as the optimization results, an energy saving of 11.7~\% is achieved for the described motion sequence compared to the initial one. Based on this, it is recommended to carry out a potential analysis in which the scalability of the approach is determined according to the number of affected industrial robots. In case of a positive evaluation, the scope of the experiment should be expanded.
\cleardoublepage
