\chapter{Einleitung}
\label{cha:Einleitung}

\section{Problemstellung}
Aus ökonomischer und ölologischer Perspektive ist die Mercedes-Benz AG  bestrebt die Energieeffizienz in der Automobilproduktion fortwährend zu minimieren. (vorerst eigene Unterstellung, ein entsprechender Beleg, bzw. eine Umformulierung folgen) Mit steigendem Automatisierungsgrad ist ein Anstieg in der Verwendung von Industrierobotern (IR) in der Automobilproduktion zu verzeichnen (vorerst eigene Unterstellung, ein entsprechender Beleg, bzw. eine Umformulierung folgen). Gemäß Stand der Technik zur Senkung der vom Roboter aufgenommenen Energie wird in Hard- und Softwarebasierte Ansätze unterschieden. Im Kontext der voliegenden Arbeit wird der Bereich Hardwareanpassung ausgeklammert. Softwareseitig werden ferner die Methoden Bahnoptimierung sowie Prozessoptimierung (Ablaufplanung bei mehreren miteinander interagierenden IR) differemziert betrachtet.\footnote{Eggers, Kai (2019): Energy-Efficient Operation of Industrial Robots, Dissertation, Maschinenbau, Hannover: Fakultät für Maschinenbau der Gottfried Wilhelm Leibniz Universität} Gegenstand der vorliegenden Arbeit ist der Bereich Bahnplanung zur Anpassung des IR Energieverbrauchs.
\section{Ausgangslage}
Unter Erstellung eines statistischen Versuchsplans (design of experiments (DOE)) konnte in einer Laborumgebung in zwei von drei Versuchen der Energieverbrauch eines KUKA KR 270 R2700 ultra IR um circa 10 \% gesenkt werden. Die Gelenkwinkel der Achsen zwei, drei und fünf sind als Freiheitsgrade des DOE definiert. Der Versuch umfasst fünf Bewegungsabläufe, wobei für jede Bewegung ein Stützpunkt auf der Hälfte der Bewegungsbahn zwischen Start und Zielpose im Programm hinterlegt ist. Der Stützpunkt ist anfänglich je Gelenk auf der Hälfte des Gelenkwinkels zwischen Startwert und Zielwert definiert. Diese Gelenkwinkelkonfiguration wird über die Versuche variiert. Der Gesamtumfang des Versuchsplans umfasst 20 randomisierte Gelenkwinkelkonfigurationen des Roboters je Stützpunkt. Für jede Konfiguration der Stützpunkte wurde der Energieverbrauch über 20 Wiederholungen aufgezeichnet. Daraus ergibt sich ein Gesamtversuchsumfang von 400 Bewegungsbahnen. Abschließend wurde der Energieverbrauch der Bewegungsbahn jedes Stützpunktes mit dem Energieverbrauch der initialen Bewegung (vor dem Hinzufügen der Stützpunkte) verglichen. Kumuliert weißen die Bewegungsbahnen der jeweils optimalen Stützpunkte einen um 10 \% geringeren Energieverbrauch als die fünf Initialbewegungen auf. Die Forschungsfrage des Proof of Concept (Poc), ob ein bestehendes Roboterprogramm durch Hinzufügen und Verschieben von Stützpunkten energetisch optimiert werden kann, wird mit ja beantwortet. Kritisch ist dabei der Versuchsumfang zu bewerten, auf dessen Grundlage die energetisch günstigen Stützpunkte identifiziert werden. Des Weiteren ist der Einfluss der Gelenkwinkelkonfiguration eines Stützpunkts nicht ohne Versuchsdurchführung  bekannt.
\section{Zielsetzung}
Die vorliegende Arbeit verfolgt das Ziel, ein bestehendes Roboterprogramm durch das Hinzufügen und Verschieben von Stützpunkten bezüglich der aufgenommenen Energie des Roboters bei Abfahren des Programms zu optimieren. Prämisse der Optimierung ist die Identifizierung der Stützpunkte a priori d. h. ohne Durchführung von Versuchen. Des Weiteren wird als Nebenbedingung definiert, dass die Verfahrzeit der optimierten Bewegungsbahn nicht signifikant von den initialen Bewegungsbahn abweichen darf.
%\section{Forschungsfrage} 
%Kann ein bestehendes Roboterprogramm durch Hinzufügen und Verschieben von Stützpunkten ohne eine signifikante Erhöhung der Verfahrzeit energetisch optimiert werden? 
\section{Geplantes Vorgehen}
% todo Projektstrukturplan	
Die Arbeit ist wie folgt gegliedert