\chapter{Einleitung}
\label{cha:Einleitung}

\section{Problemstellung}
Aus ökonomischer und ökologischer Perspektive ist die Mercedes-Benz AG fortwährend bestrebt die Energieeffizienz in der Automobilproduktion  zu minimieren. Als Kernelement der nachhaltigen Geschäftsstrategie definiert die Ambition 2039 den Weg zur $\text{CO}_2$ neutralen Mobilität unter Berücksichtigung aller \glqq Wertschöpfungsstufen des Automobils – von der Lieferkette über die Produktion bis hin zur Nutzungsphase und Entsorgung der Fahrzeuge\grqq \cite[S.~15]{Stapmanns.2022}. Industrieroboter  (IR) sind ein wesentlicher Bestandteil der Automobilproduktion und damit wichtiger Stellhebel auf dem Weg zur $\text{CO}_2$ neutralen Fertigung und Montage \cite{Maschinenbau.2023}.  Eine $\text{CO}_2$-Reduktion folgt insbesondere aus der Einsparung des Roboter Energieverbrauchs. Damit reiht sich eine Untersuchung zum  Roboter Energieverbrauch in die operative Umweltzielsetzung für die Planung und Produktion der Mercedes-Benz Werke ein \cite[S.~21]{Stapmanns.2022}. Neben dem Energieverbrauch einer Produktion ist an erster Stelle die Ausbringungsmenge und damit die Taktzeit von Relevanz. Infolgedessen lautet die Forschungsfrage der Arbeit, ob ein für die Produktion entwickeltes Roboterprogramm, Software basiert durch das Hinzufügen und Verschieben von Via-Punkten ohne eine signifikante Erhöhung der Bewegungsdauer energetisch optimiert werden? 
%Es wird unterstellt dass eine Anschaffung von Hardware zur Energieeinsparung der Roboter höhere Kosten verursacht als die Umsetzung von Software-Ansätzen. 
% 
\section{Literaturüberblick}
In der vorliegenden Bachelorarbeit wird auf die Themen mechanische Modellierung, Bahnplanung und numerische Optimierung zur Verbesserung der IR Energieeffizienz Bezug genommen. Im Folgenden wird eine Zusammenstellung der verwendeten Literatur aufgeführt. Die Übersicht erhebt keinen Anspruch auf Vollständigkeit,sondern soll vielmehr einen Einstieg in die Thematik bieten. \cite{Carabin.2017} gibt einen Überblick über bestehende Methoden zur Verbesserung der Energieeffizienz von Industrierobotern durch Anpassung der Hard- und/oder Software. Hardwareanpassungen werden in dieser Arbeit nicht untersucht. Aus Gründen der Vollständigkeit werden sie an dieser Stelle zusammenfassend dargestellt. Ein erster Ansatz basiert auf konstruktiven Änderungen an Bauteilen, z. B. dem Austausch schwerer Robotergreifer durch Leichtbaukomponenten. Durch die Reduzierung von Gewicht und Massenträgheit wird das aufzuwendende Drehmoment in den Antrieben des Roboters minimiert. Ein zweiter Ansatz ist die Installation von Systemen oder Komponenten zur Energierückgewinnung im Antriebsstrang während der Bremsphase. \cite{Pellicciari.2015} erarbeitet ein Konzept für den Austausch der zurückgewonnenen Energie über ein Gleichstromnetz (DC-Netz). Dabei wird unter anderem die DC-Netzanbindung des Roboters über eine speziell entwickelte Umrichter-Schnittstelle für den bidirektionalen Energiefluss skizziert.
Einen entscheidenden Beitrag zur Untersuchung softwaregestützter Ansätze zur Energieverbrauchsminimierung leistet \cite{Eggers.2019}.
Softwareseitig werden eine Prozess Optimierung für die Ablaufplanung interagierender IR und die Optimierung der Bahngeometrie für isoliert betrachtete Anlagen unterschieden. Der Energiebedarf Taktzeit unkritischer Anlagen kann durch eine Modifizierung der Verfahrzeit minimiert werden. Im Kontext einer Fließfertigung stellt dies einen vielversprechenden Ansatz dar. In Phasen, in denen die Anlage auf einen vor- oder nachgelagerten Prozess wartet, kann eine Optimierung des Program Override (OR) für den optimalen Energieverbrauch angewandt werden. Der Pfad des Roboters wird dabei nicht verändert. Sodass das Risiko einer Kollision nicht erhöht wird. Eine Ausnahme bilden hierbei miteinander interagierende Roboter, deren Bewegungsabläufe aufeinander abgestimmt werden müssen. Der Ausarbeitung \cite{Eggers.2019} ist hinzuzufügen, dass eine Änderung der Geschwindigkeitsprofile je nach Bewegungsart Auswirkung auf die Bahnplanung und damit den Pfad des Roboters nehmen kann. 
%
% Modell
%
Grundlage für die Optimierung des Energiebedarfs ist die Definition eines geeigneten Modells für den IR. 
Hervorzuheben ist das in \cite{Eggers.2019} beschriebene, erstmalig in \cite{Ziaukas.2017} publizierte und als Patent  \cite{Patent.2016} angemeldete Energiemodell. Im Gegensatz zu den Robotermodell-Beschreibungen in \cite{Pellicciari.2011}, \cite{Sergaki.2002} und \cite{Paryanto.2015} werden nicht nur die Verluste der elektrischen Komponenten detailliert betrachtet, sondern zusätzlich die Betriebszustände des IR unterschieden. 
Entsprechend der Phasen MOTION (Roboter in Bewegung), HOLD (Roboter im Stillstand, Antriebe sind in Regelung) und IDLE (Haltebremsen sind aktiv) wird einer präzise Analyse des Energiebedarf ermöglicht\cite{Ziaukas.2017}. 
%
Ausgangspunkt der numerischen Optimierung ist die Formulierung eines Optimierungsproblems inkl. Definition einer Zielfunktion.  
\cite{Ziaukas.2017} formuliert eine  Reduktion des Energieverbrauch durch Anpassung der Bewegungsdauer und 
%
Ein zweiter Ansatz, der in \cite{Eggers.2019} untersucht wird, ist die Optimierung der Bahngeometrie.  Im Gegensatz zu früheren Ausarbeitungen wird der Ansatz hierbei in umfangreichen praktischen Szenarien validiert. 
Entscheidend für die Durchführung der Optimierung ist die Wahl eines geeigneten Algorithmus.%Im Rahmen einer Ausarbeitung zur Optimierung von Industrierobotern für Hochgeschwindigkeitsanwendungen demonstriert \cite{Gattringer.2013} die praktische Umsetzung der in \cite{Siciliano.2011} beschriebenen System Identifikation.
%
%Optimierer
%
Auf der Grundlage des Modells wird die Bahngeometrie in \Cite{Hansen.2012} mit einem Quasi-Newton-Verfahren, siehe \cite[S.~49]{Papageorgiou.2015} minimiert. \cite{Ziaukas.2017} nutzt für das selbe Ziel einen Active-Set Algorithmus \cite[S.~445]{Luenberger.2021}. Bezüglich der theoretischen Grundlagen der Nichtlinearen-Optimierung wird auf \cite{Nocedal.2006} verwiesen. \cite[S.10~ff.]{Carabin.2017} gibt eine Übersicht weiterer, für den Anwendungsfall bereits angewandter Algorithmen.\cite{Ziaukas.2017} formuliert eine Reduktion des Energieverbrauch neben der Bewegungsdauer durch Minimierung der auftretenden Motordrehmomente. Alternativ wird in \cite{Hansen.2012} die, vom Roboter aufgenommene DC-Netzleistung als Zielfunktion herangezogen.\cite{Lin.2018} schlägt eine Mehrzieloptimierung über den  Energieverbrauch und die Verfahrdauer vor. Die häufig in den Grundlagen zitierte Ausarbeitung \cite{Saravanan.2008} betrachtet in der Optimierung simultan die Minimierung der vom Roboter aufgenommenen Leistung, den Ruck  und die Beschleunigung der einzelnen Gelenke. Abschließend sei die Ausarbeitung  \cite{Bjorkenstam.2013} genannt, welche an Stelle der nichtlinearen Optimierung die Theorie der optimalen Steuerung für eine Energieeffiziente und Kollisionsfreie Roboterbewegung zu Grunde legt.
%
%Trajektorie
%
Eine Notwendigkeit für die Optimierung einer Roboter-Bewegungsbahn ist die Definition der Bahnplanung. Hierbei werden drei wesentliche Ansätze unterschieden. In \cite{Hansen.2012} wird die Point-to-Point (PTP) Bewegungsbahn über eine B-Spline Funktion definiert. Entscheidender Nachteil dabei ist die Übertragung der Funktion auf eine industrielle Robotersteuerung aufgrund fehlenden Entwickler Schnittstellen zur Vorgabe von Sollwerten \cite[S.~55~f.]{Eggers.2019}. Eine praktikablere Umsetzung bietet die Definition und Verschiebung von zusätzlichen Via-Punkten \cite[S.~261~ff.]{Spong.2020}. Nach der Identifikation einer energieoptimierten Gelenkwinkel-Definition wird der Via-Punkt auf die Robotersteuerung übertragen. Der vom Hersteller der Robotersteuerungen implementierte Bahnplanungsansatz wird als black-box angenommen. Infolgedessen sind Abweichungen der, von der Robotersteuerung berechneten Bahn gegenüber der optimierten Bewegungsbahn möglich. \cite{Eggers.2019} vermeidet dieses Problem durch die Berechnung der zu optimierenden Bewegungsbahn auf der Originalsteuerung in einem Software-in-the-Loop (SiL) Ansatz. 
%Gleichzeitig werden damit Nebenbedingungen wie eine Drehmoment Begrenzung in der Antrieben berücksichtigt. 
%
\section{Ausgangslage}
Über die Durchführung einer statistischen Versuchsplanung (design of experiments (DOE)) im Vorfeld der vorliegenden Ausarbeitung konnte der Energieverbrauch eines KUKA KR 270 R2700 ultra IR in einer Laborumgebung um circa 10 \% gesenkt werden. Die Gelenkwinkel der Achsen zwei, drei und fünf wurden als Freiheitsgrade des DOE definiert. Der Versuch umfasst fünf hintereinander ausgeführte Bewegungsabläufe, wobei für jeden ein Via-Punkt zwischen dem Start und Zielwinkel der einzelnen Gelenkwinkel-Trajektorien festgelegt wurde. Der Via-Punkt jeder Trajektorie wurde anfänglich im Mittelpunkt zwischen Start und Zielwinkel der einzelnen Bewegungsabläufe definiert. Diese Via-Punkt Gelenkwinkel wurden mit jedem Versuch variiert. Der Gesamtumfang des Versuchsplans umfasst 20 randomisierte Konfigurationen über die drei Freiheitsgrade je Bewegungsablauf. Für jede Konfiguration wurde die Leistungsaufnahme der Antriebe in 20 Wiederholungen aufgezeichnet. Abschließend wurde für jeden Bewegungsablauf der Energieverbrauch der energieeffizientesten Via-Punkt Konfiguration mit dem Energieverbrauch der initialen Bewegungsbahn verglichen. Für die energieeffizientesten Via-Punkt Konfiguration über alle fünf Bewegungsabläufe weißt der Roboter  einen um 10 \% geringeren Energieverbrauch gegenüber der initialen Bewegungsbahnen auf. Die Forschungsfrage des DOE, ob der Roboter Energieverbrauch für ein bestehendes Roboterprogramm durch Hinzufügen und Verschieben von Via-Punkten minimiert werden kann, wurde verifiziert. Kritisch bewertet wird dabei, dass die Ursachen Grundlage der energetisch besseren Via-Punkte aus dem DOE nicht ersichtlich werden. Diese Lücke wird in der vorliegenden Bachelorarbeit geschlossen.
%
\section{Zielsetzung}
Die Zielsetzung der Arbeit definiert die Umsetzung einer Bahnoptimierung für einen Industrieroboter mit serieller Kinematik. Dafür ist ein in der Produktion eingesetztes Roboterprogramm durch das Hinzufügen und Verschieben von sog. Via-Punkten bezüglich
der aufgenommenen Energie des Roboters bei Abfahren des Programms zu optimieren. Prämisse der Arbeit ist die Identifizierung des energieoptimierten Via-Punkts auf der Grundlage eines Modells. Des Weiteren wird aufgrund strenger Taktzeitanforderungen definiert, dass die Bewegungsdauer der optimierten Bewegungsbahn nicht signifikant höher ausfallen darf als die der initialen Bewegungsbahn. Die vorliegende Arbeit verfolgt nicht das Ziel, dieselbe Modellgenauigkeit durch eine Parameteridentifikation zu erzielen wie \cite{Pellicciari.2011} und \cite{Gattringer.2013}. Von einer Nachbildung des Betriebsverhaltens der Antriebe, siehe \cite{Eggers.2019} und \cite{Ziaukas.2017} wird ebenfalls Abstand genommen. Vielmehr liegt der Fokus darauf die Grundlagen der Mechanischen Modellbildung und Optimierung darzulegen, welche in der o. g. Literatur als bekannt vorausgesetzt sind und nicht näher skizziert werden. 	
%\section{Forschungsfrage} 
\section{Geplantes Vorgehen}
% todo Projektstrukturplan	
Zur Umsetzung dieser Anforderung erfolgt im zweiten Kapitel die Beschreibung der Vorwärtskinematik sowie eine Matlab-Implementierung des rekursiven Newton-Euler-Algorithmus, auf dessen Basis die Dynamik des Roboters simuliert wird. Das dritte Kapitel beschreibt die Bahnplanung.  Basierend auf dieser Definition wird im vierten Kapitel das dynamische Modell validiert. Im fünften Kapitel erfolgt eine Beschreibung des Optimierungsproblems und die Implementierung der Via-Punkt basierten Bahnoptimierung. Anschließend werden die Ergebnisse validiert und bewertet. Die Arbeit schließt mit einer Zusammenfassung der Ergebnisse und der Skizzierung des Ausblicks. 
Die Zeitplanung der Bearbeitung ist im Anhang \ref{add:PSP} hinterlegt.