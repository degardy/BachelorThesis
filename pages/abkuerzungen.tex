% alle Abkürzungen, die in der Arbeit verwendet werden. Die Alphabetische Sortierung übernimmt Latex. Nachfolgend sind Beispiele genannt, welche nach Bedarf angepasst, gelöscht oder ergänzt werden können.

% Bei den unten stehenden Formelzeichen ist erläutert, wie explizite Sortierschlüssel über den Inhalt der eckigen Klammer angegeben werden.


%\subsection*{Allgemeine Abkürzungen} %%%%%%%%%%%%%%%%%%%%%%%%%%%%
%\nomenclature[c]{Abb.}{Abbildung}
%\nomenclature[c]{bzw.}{beziehungsweise}
%\nomenclature[c]{DHBW}{Duale Hochschule Baden-Württemberg}
%\nomenclature[c]{ebd.}{ebenda}
%\nomenclature[c]{et al.}{at alii}
%\nomenclature[c]{etc.}{et cetera}
%\nomenclature[c]{evtl.}{eventuell}
\nomenclature[b]{f.}{folgende Seite}
\nomenclature[b]{ff.}{fortfolgende Seiten}
%\nomenclature[c]{ggf.}{gegebenenfalls}
%\nomenclature[c]{Hrsg.}{Herausgeber}
%\nomenclature[c]{Tab.}{Tabelle}
%\nomenclature[c]{u. a.}{unter anderem}
%\nomenclature[c]{usw.}{und so weiter}
%\nomenclature[c]{vgl.}{vergleiche}
%\nomenclature[c]{z. B.}{zum Beispiel}
%\nomenclature[c]{}{}
% Dateiendungen %%%%%%%%%%%%%%%%%%%%%%%%%%%%%%%%%%%%
%\nomenclature[c]{EMF}{Enhanced Metafile}
%\nomenclature[c]{JPG}{Joint Photographic Experts Group}
%\nomenclature[c]{PDF}{Portable Document Format}
%\nomenclature[c]{PNG}{Portable Network Graphics}
%\nomenclature[c]{XML}{Extensible Markup Language}
\nomenclature[a]{Skaler}{Kleinbuchstabe, kursiv: $a$}
\nomenclature[a]{Vektor}{Kleinbuchstabe, fett, kursiv: $\bm{a}$}
\nomenclature[a]{Matrix}{Großbuchstabe, fett, kursiv: $\bm{A}$}
\nomenclature[a]{Punkt}{Großbuchstabe: A}

%\subsection*{Formelzeichen}
%%%%%%%%%%%%%%%%%%%%%%%%%%%%%%%%%%%
%\nomenclature[c][a]{$a$}{Beschleunigung}
%\nomenclature[c][F]{$F$}{Kraft}
%\nomenclature[c][m]{$m$}{Masse}
%\nomenclature[c][P]{$P$}{Leistung}
%\nomenclature[c][U]{$U$}{Spannung}
\nomenclature[c]{${q}_{v,i,justiert}$}{Justierter, energieoptimierter Via-Punkt}
\nomenclature[c]{$\bm{g}$}{Fallbeschleunigung}
\nomenclature[c]{$q_{i,min}$}{minimaler Gelenkwinkel des $i$-ten Gelenks}
\nomenclature[c]{$q_{i,min}$}{maximaler Gelenkwinkel des $i$-ten Gelenks}
\nomenclature[c]{$\dot{q}_{i,max}$}{maximale Gelenkwinkelgeschwindigkeit des $i$-ten Gelenks}
\nomenclature[c]{$\ddot{q}_{i,max}$}{maximale Gelenkwinkelbeschleunigung des $i$-ten Gelenks}
\nomenclature[c]{$\tau_{i,max}$}{maximaled Drehmoment des $i$-ten Gelenks}
\nomenclature[c]{$J$}{Zielfunktion}
\nomenclature[c]{$E_{mech}$}{mechanische Energie}
\nomenclature[c]{$\bm{v}^0_n$}{lineare Endeffektor-Geschwindigkeit ausgedrückt in KS$\left\{0\right\}$}
\nomenclature[c]{$\bm{\omega}^0_{0,n}$}{Endeffektor-Winkelgeschwindigkeit ausgedrückt in KS$\left\{0\right\}$}
\nomenclature[c]{$\bm{q}_s$}{Startpunkt in Gelenkkoordinaten}
\nomenclature[c]{$\bm{q}_e$}{Zielpunkt in Gelenkkoordinaten}
\nomenclature[c]{$L$}{Lagrange Funktion}
\nomenclature[c]{${\theta}$}{Gelenkwinkel}
\nomenclature[c]{$\bm{q}$}{generalisierte Koordinaten}
\nomenclature[c]{$\bm{\tau}$}{generalisierte Kräfte}
\nomenclature[c]{$\bm{J}_v$}{Lineare-Geschwindigkeits-Jacobi-Matrix}
\nomenclature[c]{$\bm{J}_{\omega}$}{Winkelgeschwindigkeits-Jacobi-Matrix}
\nomenclature[c]{K}{Kinetische Energie}
\nomenclature[c]{U}{Potenzielle Energie}
%\nomenclature[c]{$\bm{\mu}^{i}_{i}_{grav}$}{Anteil der Gewichtskraft am Drehmoment vom Verbindungsglied $i$ ausgedrückt in KS$\left\{i\right\}$}
%\nomenclature[c]{$\bm{f}^{i}_{i{grav}$}{Anteil der Gewichtskraft vom Verbindungsglied $i$ ausgedrückt in KS$\left\{i\right\}$}
\nomenclature[c]{$\bm{q}_v$}{Parametervektor des Via-Punktes in Gelenkkoordinaten}
\nomenclature[c]{$\bm{T}^{i-1}_i$}{homogene Transformationsmatrix zur Lagebeschreibung des KS$\left\{i\right\}$ ausgedrückt in KS$\left\{i-1\right\}$ }
\nomenclature[c]{$\bm{o}^0_n$}{Koordinaten des Ursprungs des Endeffektor-Koordinatensystems im Basis-Koordinatensystem}
\nomenclature[c]{$\bm{R}^n_0$}{Rotation des Endeffektor-Koordinatensystems gegenüber dem Basis-Koordinatensystem}
\nomenclature[c]{$\bm{H}$}{Endeffektor-Pose im Basis-Koordinatensystem}
\nomenclature[c]{$\theta_i$}{Gelenkwinkel}
\nomenclature[c]{$\dot{{\theta_i}}$}{Gelenkwinkel des $i$-ten Gelenks}
\nomenclature[c]{$\theta$}{Gelenkwinkel}
\nomenclature[c]{$\dot{\theta}$}{Gelenkwinkelgeschwindigkeit}
\nomenclature[c]{$\ddot{\theta}$}{Gelenkwinkelbeschleunigung}
\nomenclature[c]{$d_i$}{Gelenkabstand}
\nomenclature[c]{$a_i$}{Armelementlänge}
\nomenclature[c]{$\alpha_i$}{Verwindung}
\nomenclature[c]{KS$\left\{i\right\}$}{Koordinatensystem $i$}
\nomenclature[c]{$\bm{I}^{i}_{i}$}{Trägheitstensor des Verbindungsglieds $i$ ausgedrückt in KS$\left\{i\right\}$}
\nomenclature[c]{$\bm{I}^{0}_{i}$}{Trägheitstensor des Verbindungsglieds $i$ ausgedrückt in KS$\left\{0\right\}$}
\nomenclature[c]{$\bm{M}(\bm{q})$}{Massenmatrix}
\nomenclature[c]{$\bm{r}^0_{C_i}$}{Lage des Masseschwerpunkts vom Verbindungsglied $i$ im  KS$\left\{0\right\}$}
\nomenclature[c]{$\bm{g}$}{Fallbeschleunigung}
\nomenclature[c]{$m_i$}{Masse des Körpers $i$}
\nomenclature[c]{$\bm{C}(\bm{q},\dot{\bm{q}})$}{Anteil der Corioliskrft und Zentrifugalkraft an den generalisierten Kräften}
\nomenclature[c]{$\bm{g}(\bm{q}$)}{Anteil der Schwerkraft  an den generalisierten Kräften}
\nomenclature[c]{$ \boldsymbol{f}_i $}{Kraft, die vom Verbindungsglied $ i-1 $ auf das Verbindungsglied $ i $ ausgeübt wird}
\nomenclature[c]{$ -\boldsymbol{f}_{i+1} $}{Kraft, die vom Verbindungsglied $ i+1 $ auf das Verbindungsglied $ i $ ausgeübt wird}
\nomenclature[c]{$ \boldsymbol{\mu}_i $}{Drehmoment, welches vom Verbindungsglied $ i-1 $ auf das Verbindungsglied $ i $  ausgeübt wird}
\nomenclature[c]{$ -\boldsymbol{\mu}_{i+1} $}{Drehmoment, welches vom Verbindungsglied $ i+1 $ auf das Verbindungsglied $ i $ ausgeübt wird}
\nomenclature[c]{$ \boldsymbol{f}_i $}{Kraft, die vom Verbindungsglied $ i-1 $ auf das Verbindungsglied $ i $ ausgeübt wird, ausgedrückt in KS$\left\{i\right\}$}
\nomenclature[c]{$ -\boldsymbol{f}^{i+1}_{i+1} $}{Kraft, die vom Verbindungsglied $ i+1 $ auf das Verbindungsglied $ i $ ausgeübt wird,  ausgedrückt in KS$\left\{i+1\right\}$}
\nomenclature[c]{$ \boldsymbol{\mu}^{i}_i $}{Drehmoment, welches vom Verbindungsglied $ i-1 $ auf das Verbindungsglied $ i $  ausgeübt wird,  ausgedrückt in KS$\left\{i\right\}$}
\nomenclature[c]{$ -\boldsymbol{\mu}^{i+1}_{i+1} $}{Drehmoment, welches vom Verbindungsglied $ i+1 $ auf das Verbindungsglied $ i $ ausgeübt wird,  ausgedrückt in KS$\left\{i+1\right\}$}
\nomenclature[c]{$ \boldsymbol{r}^{i}_{i-1,C_i} $}{Vektor vom Ursprung des KS$\left\{i-1\right\}$ zum Masseschwerpunkt $ C_i $ des Verbindungsglieds $i$ ausgedrückt in KS$\left\{i\right\}$ }
\nomenclature[c]{$ \boldsymbol{r}^{i}_{i,C_i} $}{Vektor vom Ursprung des KS$\left\{i\right\}$ zum Masseschwerpunkt $ C_i $ ausgedrückt in KS$\left\{i\right\}$}
\nomenclature[c]{$ \boldsymbol{r}^{i}_{i-1,i} $}{Vektor vom Ursprung des KS$\left\{i-1\right\}$ zum Ursprung des  KS$\left\{i\right\}$ ausgedrückt in KS$\left\{i\right\}$ }
\nomenclature[c]{$ i_i$}{Übersetzungsverhältnis des Getriebes $i$}
\nomenclature[c]{ $ \dot{\boldsymbol{p}}_{C_i} $}{ Lineare Geschwindigkeit im Masseschwerpunkt $ C_i $}
\nomenclature[c]{ $ \dot{\boldsymbol{p}}_i $}{Lineare Geschwindigkeit im Ursprung des KS$\left\{i\right\}$}
\nomenclature[c]{ $ \boldsymbol{\omega}_i $}{Winkelgeschwindigkeit des Verbindungsglieds $i$}
\nomenclature[c]{ $ \ddot{\boldsymbol{p}}_{C_i} $}{Lineare Beschleunigung im Masseschwerpunkt $ C_i $}
\nomenclature[c]{ $ \ddot{\boldsymbol{p}}_i $}{Lineare Beschleunigung im Ursprung des KS$\left\{i\right\}$}
\nomenclature[c]{ $ \boldsymbol{\dot{\omega}}_i $}{Winkelbeschleunigung des Verbindungsglieds $i$}
\nomenclature[c]{ $ \dot{\boldsymbol{p}}^i_{C_i} $}{ Lineare Geschwindigkeit im Masseschwerpunkt $ C_i $ ausgedrückt in KS$\left\{i\right\}$ } 
\nomenclature[c]{ $ \dot{\boldsymbol{p}}^i_i $}{Lineare Geschwindigkeit im Ursprung des KS$\left\{i\right\}$ ausgedrückt in KS$\left\{i\right\}$}
\nomenclature[c]{ $ \boldsymbol{\omega}^i_i $}{Winkelgeschwindigkeit des Verbindungsglieds $i$ ausgedrückt in KS$\left\{i\right\}$}
\nomenclature[c]{ $ \ddot{\boldsymbol{p}}^i_{C_i} $}{Lineare Beschleunigung im Masseschwerpunkt $ C_i $ ausgedrückt in KS$\left\{i\right\}$}
\nomenclature[c]{ $ \ddot{\boldsymbol{p}}^i_i $}{Lineare Beschleunigung im Ursprung des KS$\left\{i\right\}$ ausgedrückt in KS$\left\{i\right\}$}
\nomenclature[c]{ $ \boldsymbol{\dot{\omega}}^i_i $}{Winkelbeschleunigung des Verbindungsglieds $i$ ausgedrückt in KS$\left\{i\right\}$}
\nomenclature[c]{$t$}{Zeit}
\nomenclature[c]{$t_s$}{Startzeit der Bahnbewegung}
\nomenclature[c]{$t_v$}{Zeitpunkt für dasn Erreichen des Via-Punkts}
\nomenclature[c]{$t_e$}{Endzeitpunkt der Bahnbewegung}
%\nomenclature[c]{$a$}{Parameter der Bahnplanung für i = 0,...,6}
\nomenclature[c]{$P_{i}$}{Mechanische Leistung für den Antrieb des $i$ten Gelenks}
\nomenclature[c]{$P_{mech}$}{Mechanische Leistung}
\nomenclature[c]{${q}$}{Gelenkwinkel}
\nomenclature[c]{$\dot{q}$}{Gelenkwinkelgeschwindigkeit}
\nomenclature[c]{$\ddot{q}$}{Gelenkwinkelbeschleunigung}
\nomenclature[c]{$\bm{E}$}{Einheitsmatrix}
\nomenclature[c]{$\bm{S}$}{Schiefsymmetrische Matrix}
\nomenclature[d]{IR}{Industrieroboter}
\nomenclature[d]{$\text{CO}_2$}{Kohlenstoffdioxid }
\nomenclature[d]{DC-Net}{Gleichstromnetz}
\nomenclature[d]{OR}{Override}
\nomenclature[d]{PTP}{Point-to-Point}
\nomenclature[d]{DOE}{Design of Experiments}
\nomenclature[d]{DH}{Denavit-Hartenberg}
\nomenclature[d]{RNEA}{Rekursiver-Newton-Euler-Algorithmus}
\nomenclature[d]{Pose}{Position und Orientierung}
\nomenclature[d]{KR C}{KUKA Robot Control}
\nomenclature[d]{CAD}{Computer Aided Design}
\nomenclature[d]{PMSM}{permanentmagneterregte Synchronmaschinen}
\nomenclature[d]{ESB}{Ersatzschaltbild}
\nomenclature[d]{Home}{Grundstellung}
\nomenclature[d]{TP-Filter}{Tiefpassfilter}
\nomenclature[d]{RSI}{RobotSensorInterface}
\nomenclature[d]{UDP}{User Datagram Protocol}
\nomenclature[d]{API}{Application Programming Interface}
\nomenclature[d]{Winsock}{Windows Sockets 2}
\nomenclature[d]{KRL}{KUKA Robot Language}
\nomenclature[d]{ASCII}{American Standard Code for Information Interchange}
\nomenclature[d]{JSON}{JavaScript Object Notation}
\nomenclature[d]{TCP}{Tool-Center-Point}
\nomenclature[d]{SiL}{Software-in-the-Loop}
\nomenclature[d]{SQP}{sequential quadratic programming}
\nomenclature[d]{KKT-Konvergenzkriterien}{Karush-Kuhn-Tucker-Konvergenzkriterien}
\nomenclature[d]{src}{source-file}
\nomenclature[d]{dat}{data-file}
